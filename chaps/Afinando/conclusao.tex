# TALVEZ TRAZER PARA CÁ COMPARAÇÃO DA EDITATONA COM O BLOG DA TURMA


Com esta nova metodologia, percebemos que os templates criados para auxiliar na gestão das páginas de testes deixaram de ser tão úteis para a editatona. Enquanto com os editores juntos presencialmente na mesma máquina eles se mostravam um excelente caminho para a criação de rascunhos e familiarização com o MediaWiki, a dinâmica à distância inviabilizou sua utilização na prática.


Para abrir este capítulo de conclusão, antes de nos debruçarmos a uma tentativa de nos aproximarmos a uma resposta para a pergunta que enuncia este trabalho, entendemos ser necessário nos voltarmos para a nossa prática de escrita de científica.

"Numa enciclopédia não “está” o conhecimento da humanidade, pois conhecimento “não está”, sendo dinâmica interminável de desconstrução/reconstrução. Está aí apenas a compilação dos processos estabilizados de produção de conhecimento, no fundo, já congelados como informação. Poder-se-ia, entretanto, alimentar outra concepção de enciclopédia como dinâmica aberta de produção de conhecimento e que subsiste em discussão permanente, através de processos de edição interminável, sem estabilização à vista. O que mais se aproxima disso é a wikipedia, ainda que sua configuração metodológica se oriente por outra visão (modernista). Restaria discutir ainda se “todos podem editar”" (DEMO, 2009)

# Falar o quanto no desenvolvimento deste trabalho foi ficando cada vez mais claro o quanto os dois processos são parecidos. Destacar a escolha do ferramental CTS e algumas de suas características.

## Para ser mesmo aderente a ciência aberta eu deveria publicar as respostas das entrevistas também, mas isso pode dar mais munição para críticas ao meu trabalho.
O quanto no ringue da ciência a abertura total de seus dados e resultados parciais pode ser visto como uma forma de abaixar a guarda?

O texto científico tem como missão convencer o leitor de que seu conteúdo é verdadeiro, e a forma tradicional de se fazer isso é se afirmando como o revelador de uma verdade natural que existia a priori e apenas aguardar por ser descoberta e revelada. A compreensão Latouriana do fazer ciência como um processo de traduções, negociações e estabilizações localizadas no tempo e no espaço inverte a expectativa do leitor pela guarda levantada do texto. Um leitor CTS não engolirá argumentados naturalizados em revelações óbvias, e demandará do autor a transparência e abertura que, se por um lado podem o tornar mais frágil para a crítica de modernos, o tornará mais robusto perante aqueles que entenderem a ciência como um processo.

É curioso notar que o mesmo efeito pode ser espelhado para a Wikipédia. Ao adotar um processo editorial aberto a enciclopédia se torna mais criticável por tradicionalistas, mas como já explorado aqui no capítulo 3, é exatamente nesta caraterística que ela baseada toda sua auditabilidade e consequente qualidade.

Podemos então concluir que assim como o fazer enciclopédico, o fazer ciência também sofre um deslocamento com a mudança deste paradigma moderno de verdades reveladas. A explicitação de seus desvios e composições passa a não só não ser vista como um problema, como passa a ser cobrada pelo leitor, que acostumado com a possibilidade de seguir rastros para auditar (mesmo que em alguns casos isso seja praticamente impossível, como em Latour...) a informação recebida até um nível mais profundo que o próprio conteúdo apresentado a ele, não mais se contentará com pregadores de verdades estanques e absolutas.

# CITAR NO PARÊNTESIS CIÊNCIA EM AÇÃO - LABORATÓRIOS

# Nesse texto da introdução articular algo sobre termos criado nosso contra laboratório:

Usar essas reflexões na conclusão, dizendo que utilizamos em parte da pesquisa as "centrais de cálculo" que são "pontos de passagem obrigatória", mas também "criamos nosso contra laboratório", fazendo nossos próprios scripts e páginas de cruzamentos de dados.

Falar em algum momento que a equipe de Analytics criou uma central de cálculo, que tem o espaço no meta e o portal de estatísticas. Serve para medir e acompanhar à distância comunidades, grants, eventos e "impactos", com indicadores que se tornam globais nos projetos apoiados pela WMF.

Citar Latour: (citação já usada no cap 4 WEP. Ver que partes usar para não ficar pesada a repetição)

""[...] construir centros implica trazer para eles elementos distantes – permitir que os centros dominem à distância –, mas sem trazê-los "de verdade" [...] Esse paradoxo é resolvido criando-se inscrições que conservarão, simultaneamente, o mínimo e o máximo possível, através do aumento da mobilidade, da estabilidade ou da permutabilidade desses elementos. Esse meio termo entre presença e ausência muitas vezes é chamado de informação. Quando se tem uma informação em mãos, tem-se a forma de alguma coisa sem ter a coisa em si. Como sabemos, essas informações (ou formas, ou formulários, ou inscrições – todas essas expressões designam o mesmo movimento e resolvem o mesmo paradoxo) podem ser acumuladas e combinadas nos centros "(LATOUR, 2000, p. 396)" p. 17

\section{Afinal, todos podem editar?}

O presente estudo não se propõe a desestimular a participação de novatos e nem de deslegitimar os esforços inclusivos do Movimento Wikimedia, mas de colocar em perspectiva o discurso generalizante de que todos podem participar, para que, as participações possam, entendendo o contexto de diferentes relações de tipos de usuários com a enciclopédia, ser realizadas de forma mais produtiva, a partir da compreensão do contexto em que acontecem.

# Usar o termo "desobiviar" no parágrafo anterior.

# Resgatar a regra do 1% e como as decisões estão concentradas nas mãos de poucos.

# Falar sobre pilares da Wikipédia e a dicotomia entre a atividade intensa em alguns espaços de governança no domínio Wikipédia para pequenas decisões e a grande atividade das ferramentas automatizadas tomando decisões sem nenhuma discussão (e com pequena discussão prévia em seus espaços).
# Apresentar números consolidados sobre ação das barreiras automatizadas comparadas com edições de humanos.

# Falar sobre o desconhecimento de administradores sobre algumas das ferramentas que criar essas barreiras.

# Resgatar falas dos participantes de nossas atividades, das dificuldades encontradas e de seu desejo de continuar editando.
"Qualquer um pode inserir ali qualquer frase, mas apenas um conjunto restrito de alegações irá sobreviver. Uma afirmativa ali não tem qualquer valor intrínseco; seu destino – se ela será acatada, rejeitada, atenuada ou reforçada – depende do que dela farão os demais wikipedistas, humanos ou não humanos. Se a realidade é o que resiste, como postulou Latour (1987), a estabilização de um verbete pode ser entendida nos termos da resistência de suas proposições às intervenções dos wikipedistas (ESTEVES; CUKIERMAN, 2011)" (ESTEVES, 2010. p. 90).

"O problema é que se escondem as contradições deste tipo de dinâmica: de um lado, instiga-se que todos participem; de outro, desconfia-se que, onde todos participam, o resultado pode ser frívolo; para evitar isso usam-se dois discursos incompatíveis: participar à vontade, mas respeitando regras cada vez mais rígidas." (DEMO, 2009)

"a história da wikipédia indica que os procedimentos estão se tornando a própria finalidade maior, a ponto de sabotar um dos pontos de partida mais iluminados: os textos são sempre abertos. Buscam-se mentes indomáveis, desde que aceitem ser domadas no processo." (DEMO, 2009)

"inovações espetaculares e traições comuns em práticas libertárias que convivem, ironicamente, com autoridades indiscutíveis. Depois de oito anos, o mote “todos podem editar livremente” já tem validade muito relativa; para muitos já sequer vale, tamanhas são as regras impostas entrementes a quem quer ser editor. Mesmo assim, isto não destrói a beleza do projeto, embora revele, à revelia, o drama da liberdade de expressão, essencial para o conhecimento questionador: desde que a rebeldia tenha alguma proposta concreta, ao pôr-se a realizá-la, deixa de ser rebelde; de fato, quem propõe mudanças, não as pode gerir! Toda proposta crítica, ao instituir-se, vira paradigma e, como tal, decai para a história que passa e ultrapassa." (DEMO, 2009)

Posso usar a passagem a seguir do Zé para articular algo sobre como a representatividade dos que participam dos espaços de tomada de decisão é reconhecida e referendada pela comunidade, permitindo assim a estabilização de fatos. Porém, fica em aberto uma pergunta sobre quais as fronteiras e abrangências desta representatividade. Para uma enclicopédia que se propõe ter "toda a soma do conhecimento humano, possivelmente esse ator-rede não está enrredando boa parte da população global, que transboda. Mas vale a reflexão, seria realmente possível uma rede de tamanha abrangência?

# "Michel Callon ressalta que as controvérsias que determinam a estabilização (ou não) de um fato ou de um artefato, estão diretamente relacionadas com a questão da representatividade efetiva de determinados grupos por seus porta-vozes (CALLON, 1986, p. 15)." (zé marcos)

\section{Efeitos colaterais}

De forma genérica abrir falando que a pesquisa gerou subprodutos que trazem os seguintes benefícios:
    • Auxiliar futuros editores novatos a terem uma vida mais fácil.
    • Disponibilizar ferramentas e bases de dados estruturadas sobre editatonas para outros pesquisadores.
    • Gerar conteúdo disponibilizado livremente sobre assuntos relacionados às pesquisas de Informática e Sociedade.
    
# Falar sobre a crescente área de mapeamento de controvérsias e do painel de BI criado para outros pesquisadores utilizarem
## ver cartografia iniciada por Bernado em
http://climanawikipedia.blogspot.com.br/ como uma possíveis inspiração para painel de BI

## Por aqui citar Contropedia, que também serviu de inspiração inicial para a pesquisa.

# Compartilhar códigos-fonte para extração de dados sobre atividade de governança nas Wikipédias. (Estará nos anexos)
# Apresentar números de acesso dos artigos trabalhados e falar sobre visibilidade dos temas relacionados às demais pesquisas do laboratório.

## Apresentar projeção de acessos em Y tempo.

## Atualizar número de acessos aos verbetes do trabalho do Esocite e o utilizar como possível referência para projeção matemática.

## Quando falar do efeito colateral de criação de conteúdo sobre nossos assuntos destacar que mais de 60\% dos acessos da Wikipédia vem de ferramentas de buscas[[citation needed]], e quanto mais palavras tiver mais fácil será de um verbete ser apresentado entre os primeiros resultados.

# Compartilhar roteiro detalhado e válido sobre como organizar uma editatona virtual. (que estará nos anexos) ## criar learning pattern sobre isso

## Criar Learning Patterns sobre os templates para páginas de testes e citar como contribuição adicional da pesquisa.

# falar do citation hunt

\section{Trabalhos futuros}

# Acompanhar densamente a criação de uma regra polêmica automatizável, seguindo todo seu caminho até sua codificação em software.

## Narrar a abertura da caixa-preta de uma regra. Exemplos: captcha na ptwiki e visualeditor globalmente.

# Narrar densamente o histórico da criação de filtros de edição.

# Atuação dos bots reversores e sua interação especialmente com novatos.

# Governança delegada: aprofundar na comparação entre a atividade densa de discussão em assuntos mais "simples" enquanto outros que poderiam ser mais "centrais" ficam nas mãos de softwares e passam despercebidos pela comunidade.
# Diferenças da influência de barreiras automatizadas em diferentes idiomas.

Outro tópico de interesse que permeia este estudo é a diferença entre os idiomas da Wikipédia e suas comunidades. Se a busca por ferramentas automatizadas para tomada de decisão já é complexa e questionável per se, a utilização desse ferramental para tratar de enciclopédias com regras editoriais distintas e editores de locais, culturas e formações também tão diversas se torna um desafio ainda maior. Em sua busca por ferramentas que pudessem ser utilizadas em todos os idiomas a comunidade Wikimedia percebe que, como diz Ivan da Costa Marques (2016, p.3-4), "a divisão local x global não nos serve. Ela mais obscurece do que ilumina nossas opções porque esse global é o local de uma determinada rede e o que ela nos apresenta como o universal é o particular no poder."

# Uso de machine learning para avaliação de qualidade.

## Trabalhar trecho da entrevista com Aaron onde ele cogita tirar a palavra "Objective" do nome do ORES. (posso buscar algumas inspirações a mais nos tópicos do capítulo 4 da qualificação)

# ores e avaliacao qualitativa.

# comparar editatona presencial e virtual.

\section{Texto da qualificação que pode ser usado em algum lugar}

Para falar sobre o uso de MAMs na Wikipédia o trabalho se afasta um pouco da materialidade da escrita enciclopédica e busca entender o conceito de modelo de aprendizagem de máquina, seu discurso de objetividade e neutralidade e as controvérsias em torno de sua definição e utilização cotidiana. Em meio a discursos descomedidos que apontam modelos de aprendizagem de máquina como a nova metodologia revolucionária que potencializa ganhos, otimiza custos e oferece um tratamento neutro e igualitário para todos, podemos encontrar vozes dissidentes, como Cathy O'Neil (2017), Ph.D. em matemática pela Universidade de Harvard, que afirma que os modelo carregam "opiniões codificadas (...) que automatizam o status quo". (O'Neil, 2017)

# trazer aqui também uma citação ao livro dela de (O'Neil, 2016).

Partimos então para uma explicação mais detalhada sobre como esses modelos são treinados e como sua precisão é medida. O conceito de ground truth , a "verdade fundamental" dos Modelos de Aprendizagem de Máquina que serve como gabarito tanto para o treino como para a avaliação dos modelos, é então apresentado e sua caixa-preta é aberta. Quando um desenvolvedor afirma que um modelo acerta em "x\% dos casos", ele está confiando que sua ground-truth está certa em 100\% dos casos. Caso existam erros na ground truth eles são considerados problemas a serem corrigidos, pois os melhores modelos são treinados sempre com as melhores anotações que, se feitas de forma clara e objetiva, poderão ser consideradas porta vozes da verdade. Esta definição é problemática para alguns pesquisadores, como Bart Lamiroy, que afirma "ground truth é intrinsecamente ambígua, e é, no máximo, uma representação de um contexto de interpretação não formalizado, eventualmente conectado a algum nível de interpretação humana" (2013 p.157), e propõe a seus colegas de computação a troca do termo "ground truth" por "referential interpretation". Está visão de Limoroy é então utilizada para discutir se MAMs podem ser compatíveis com a cultura de construção coletiva de conhecimento das comunidades wiki.

# citar por aqui (SARABADANI et al., 2017) para falar da construção de mais modelos de aprendizagem na comunidade Wikimedia.