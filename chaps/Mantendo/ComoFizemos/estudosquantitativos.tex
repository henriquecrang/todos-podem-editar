
\subsection{Estudos quantitativos e a Wikipédia}

Metodologias quantitativas podem entrar em rota de colisão com os Estudos CTS, pois tendem a realizar inferências que confiam em metadados como fiéis porta-vozes dos fenômenos estudados, assim como tendem a dedicar pouca atenção às traduções\footnote{Utilizamos o termo ``tradução'' da mesma forma que  Bruno Latour, apoiados no conceito criado por Michel Serres em ``\textit{La traduction}'', de 1974, e retrabalhado em ``\textit{Le tiers-instruit}'', de 1991.}, traições, enquadramentos e transbordamentos que estão necessariamente envolvidos no processo de criação de camadas de conhecimento. Como proposto por Henrique Cukierman, em palestra no evento ``Avaliação da produção científica brasileira: pensando com a história das ciências'', organizado em 2011 pela SBHC e pelo HCTE\footnote{Sociedade Brasileira de História da Ciência (SBHC) e Programa de Pós-Graduação em História das Ciências e das Técnicas e Epistemologia da Universidade Federal do Rio de Janeiro (HCTE).}, sobre a confiança nos números, geralmente traduzida como ``objetividade'': ``\textit{O que há de especial com a linguagem da quantidade? Respondendo de forma sucinta, pode-se dizer que a quantificação é uma tecnologia de controle a distância, pouco ou nada relacionada com a chamada 'verdade' da natureza}'' (\cite[p.5]{cukierman_uma_2011}). Em seguida, citando Theodore M. Porter, ressaltou que ``\textit{a quantificação é parte de uma estratégia de intervenção, e não de mera descrição}'' (\cite[43]{porter_1996}, apud \cite[p.5]{cukierman_uma_2011}).

Na literatura de estudos sobre controvérsias na Wikipédia é comum encontrar pesquisadores/as caindo nesta armadilha da autoproclamada "objetividade" dos dados, realizando estudos profundos unicamente alicerçados em metadados que são assumidos como verdades puras e neutras, sem que os/as pesquisadores/as se ocupem em compreender como são elaborados, em observar quais ações foram de fato realizadas pelos/as usuário/as que viriam a gerar tais metadados estudados. É o caso sintetizado por Latour (\citeyear{latour_ciencia_1987}) como "\textit{o representante sendo assumido automaticamente como o representado}". 

Sigamos então com uma breve revisão bibliográfica para observar a materialidade desta prática de utilização dos metadados como porta vozes suficientes nos estudos mais citados sobre controvérsias na Wikipédia.

Em um trabalho bastante citado\footnote{89 citações segundo o Google Scholar. Página disponível em https://scholar.google.com.br/scholar?hl=pt-BR\&as\_sdt=0\%2C5\&q=Edit+wars+in+Wikipedia\&btnG= , acessada em 25 de março de 2020.}, “\textit{Edit wars in Wikipedia}”, Sumi et al (2011) propõem o índice M de controvérsia, observando “\textit{como' o número de edições e reversões desviam em algumas páginas dos números seguidos pela maioria dos artigos}”\footnote{Todas as citações desta revisão bibliográfica foram traduzidas livremente pelo autor.}. Mapeando duplas de editores/as que se revertem em um artigo e seu volume de edições no mesmo texto, os autores testam seu modelo em 6 diferentes idiomas e observam que menos de 1\% dos artigos apresentam um resultado significativo em seu índice.

O trabalho anterior foi a base do estudo ``\textit{The most controversial topics in Wikipedia: A multilingual and geographical analysis}'' de Yaseri e aliados, citado já 100 vezes por outros estudos.\footnote{https://scholar.google.com.br/scholar?hl=pt-BR\&as\_sdt=0\%2C5\&q=The+most+controversial+topics+in+Wikipedia\%3A+A+multilingual+and+geographical+analysis\&btnG= , acessada em 25 de março de 2020.} Nele foi utilizado o 
índice M como ponto de partida para observar versões em diferentes idiomas da enciclopédia buscando aproximações e similaridades entre grupos de idiomas próximos, com o objetivo explícito de criar ``\textit{um indicador multilíngue e independente da cultura}'' (\cite{yasseri_controversial_2014}). Em sua pesquisa, eles identificam que a maior parte das controvérsias são localizadas\footnote{Utilizamos ao longo desta pesquisa o termo ``localizada'' da mesma forma que o Movimento Wikimedia, adjetivando coisas que não sejam definidas globalmente pelo movimento, e podem ser instanciadas localmente de forma distinta pelas comunidades de cada projeto.}, e não se repetem tão comumente em outros idiomas (menos ainda se forem de grupos linguísticos diferentes), o que reforça a dificuldade de utilizar metadados globais para mapear comunidades com práticas distintas.

Já Vuong et al. (\citeyear{vuong_ranking_2008}), em “\textit{On ranking controversies in wikipedia: models and evaluation}”, com 115 citações\footnote{	https://scholar.google.com.br/scholar?hl=pt-BR\&as\_sdt=0\%2C5\&q=On+ranking+controversies+in+wikipedia\%3A\+models+and+evaluation\%E2\%80\%9D\&btnG= , acessada em 25 de março de 2020.}, resolvem levar em consideração não somente a atividade editorial em um artigo, como igualmente o perfil dos/as usuários/as envolvidos/as. Utilizando também a idade do artigo (em número de revisões salvas, e não em tempo corrido desde sua criação), os autores propõem diferentes índices partindo da seguinte premissa: “\textit{um/a usuário/a é mais controverso/a se participa de disputas em artigos menos controversos e um artigo é mais controverso se nele participam de disputas usuários/as menos controversos/as}”. Ambos os conceitos de controvérsia que se retroalimentam, tanto de usuários/as como de artigos, emergem de metadados brutos.

Em “\textit{There is no deadline: time evolution of Wikipedia discussions}”, o artigo menos citado desta revisão bibliográfica mas ainda assim com relevantes 32 menções\footnote{https://scholar.google.com.br/scholar?hl=pt-BR\&as\_sdt=0\%2C5\&q=There\+is+no+deadline\%3A+time+evolution+of+Wikipedia+discussions\&btnG= , acessada em 25 de março de 2020.}, Kaltenbrunner e Laniado (\citeyear{laniado_emotions_2012}) voltam-se para os metadados de edição das páginas de discussão dos artigos, onde inicialmente não encontraram padrões de volume de edições correlacionados com os metadados de suas páginas correspondentes no domínio principal. Aprofundando a inferência, criaram o indicador\textit{h}, que aponta o maior número de discussões que seja igual ao número mínimo de respostas nelas. Observando o tempo que um artigo leva para incrementar seu \textit{h}, o estudo procura “\textit{identificar escaladas ou estabilizações de controvérsias}” de forma totalmente automatizada.

Em “\textit{Visual analysis of controversy in user-generated encyclopedias}”, citado 97 vezes\footnote{	https://scholar.google.com.br/scholar?hl=pt-BR\&as\_sdt=0\%2C5\&q=Visual+analysis+of+controversy+in+user-generated+encyclopedias\&btnG= , acessada em 25 de março de 2020.}, Brandes e Lerner (\citeyear{brandes_visual_2008}) estão interessados em saber quem edita depois de quem, e qual a diferença de tempo entre essas edições. Com essas informações, o artigo cria uma visualização de usuários/as agrupados/as, simbolizando “facções” em disputas de edições. Neste caso, após gerar seus mapas a partir dos metadados, os pesquisadores buscaram ler os artigos e estudar o perfil dos/as usuários/as envolvidos/as nas disputas, tanto na versão inglesa da Wikipédia como na alemã. De forma não surpreendente para nós, após está análise qualitativa, concluíram que seu modelo tende a aproximar vândalos\footnote{Termo utilizado no Movimento Wikimedia para se referir a usuários/as que fazem edições de má fé.} e combatentes de vandalismo, pois “\textit{ambos podem apresentar o comportamento ‘um contra todos}’” (\cite{brandes_visual_2008}). Pensando em melhorias de seu trabalho que deem conta de fazer de forma automatizada tão importante distinção entre esses perfis de usuário/a tão claramente desiguais, propõem que sejam adicionados futuramente à equação mais metadados, tais como registros de bloqueios dos/as usuários/as e conversas realizadas nas páginas dos/as usuários/as.

Por fim, citamos o badalado\footnote{137 citações mapeadas no Google Scholar. Disponível em https://scholar.google.com.br/scholar?hl=pt-BR\&as\_sdt=0\%2C5\&q=Global+disease+monitoring+and+forecasting+with+Wikipedia\&btnG= , acessada em 25 de março de 2020.} trabalho “\textit{Global disease monitoring and forecasting with Wikipedia}”, de Generous et al (\citeyear{generous_global_2014}), que foi objeto de grande cobertura da mídia\footnote{Em uma rápida busca no Google ainda hoje é fácil recuperar matérias de veículos como Washington Post https://www.washingtonpost.com/news/to-your-health/wp/2014/11/13/how-wikipedia-reading-habits-can-successfully-predict-the-spread-of-disease/ , LA Times https://www.latimes.com/science/sciencenow/la-sci-sn-wikipedia-flu-disease-predictor-20141113-story.html e Revista Galileu https://revistagalileu.globo.com/Ciencia/Saude/noticia/2014/11/como-wikipedia-pode-ajudar-monitorar-doencas.html} ao buscar identificar, antes das autoridades de saúde pública, surtos de doenças epidêmicas a partir do comportamento dos/as usuários/as nas Wikipédias. Mesmo se esforçando em realizar calibragens locais, fica claro em seus resultados que enquanto muito bem-sucedidos (como alardeado pela imprensa) para algumas doenças em alguns países em um determinado recorte temporal, o modelo puramente baseado em metadados fracassou na maioria dos casos testados.

Como ficou claro, a tradição de utilizar ferramentas de ciência de dados e análise de metadados em pesquisas costuma caminhar em uma proposta metodológica antagônica à concepção CTS de abertura de caixas-pretas aqui defendida. O impasse apresentado em tentar simultaneamente acompanhar densamente dinâmicas locais e utilizar técnicas quantitativas generalizantes pode parecer então incompatível com os Estudos CTS.

Porém, nem tudo está perdido. Mesmo com alguma ressalva, Latour (\citeyear[p.167]{latour_cogitamus_2010}) relata em Cogitamus que “\textit{sim, reconheço, as ferramentas digitais são um veneno. Mas, talvez, também ofereçam um remédio. Ao menos, isso é o que exploro há dez anos com os alunos dos cursos chamados ‘mapeamento de controvérsias’. Talvez fosse possível aprender a se orientar nas disputas, com a condição de contar com uma plataforma suficientemente calibrada e padronizada, para dar a um público virtual – ainda a ser inventado – hábitos comuns}". Seguindo esta visão, podemos observar “\textit{As controvérsias da ciência na Wikipédia em português: o caso do aquecimento global}”, tese de doutorado defendida em 2014 por Bernardo Esteves. Em um grande esforço para estudar as controvérsias sobre mudanças climáticas utilizando ferramental metodológico CTS, Esteves (\citeyear[p.295]{esteves_as_2014}) observa que “\textit{na Wikipédia, cada ação dos editores deixa rastros disponíveis para consulta de pesquisadores e demais interessados, abrindo uma janela para um repositório riquíssimo de informações sobre como os usuários negociam suas versões de verdade}”, e, para dar conta desses rastros, acompanha editores/as, regras editoriais, robôs, bloqueios, discussões e proteções de páginas, narrando processos de resolução de controvérsias e de estabililização de textos enciclopédicos. Em sua pesquisa, Esteves fez tanto análises quantitativas como qualitativas, e em seus esforços de construir um olhar ao mesmo tempo amplo como o da águia e míope como o da formiga, chegou a cogitar a criação de um indicador de controvérsias que pudesse ser utilizado na Wikipédia em português imbricando estes olhares.\footnote{"\textit{Acreditamos que os resultados deste estudo poderiam ganhar mais refinamento e resolução caso tivéssemos explorado mais a fundo as ferramentas computacionais para extrair e tratar dados disponíveis na base de dados da Wikipédia. Ademais, os resultados de um tratamento estatístico mais robusto poderiam talvez contribuir para o desenvolvimento de um índice de controvérsia mais adequado às especificidades das interações ente os usuários da Wikipédia lusófona}" \cite[p.296]{esteves_as_2014}.}

Seguindo ainda outro exemplo da linha de “\textit{Mapping Controversies}” apresentada na citação acima de Latour, o Médialab do Instituto de Estudos Políticos de Paris (Science Po), o mesmo Instituto onde Bruno Latour trabalha, foi enredado com a Fundação Barcelona Media, ferramentas de visualização de dados, o DesityDesign Lab do Politécnico de Milano, metadados da Wikipédia, o Digital Methods Initiave da Universidade de Amsterdam e orçamentos de pesquisa da Comissão Europeia no Projeto Contropedia, em um esforço para “\textit{construir uma plataforma de visualização e análises em tempo real de controvérsias na Wikipédia”} (\cite{noauthor_site_2013})\footnote{Todas as citações de textos da Contropedia foram traduzidas livremente pelo autor.}. Criado em novembro de 2013, o projeto é brevemente mencionado por Esteves em sua tese, mas no momento de seu estudo praticamente não havia informações publicadas sobre o desenvolvimento do projeto. Agora, o projeto já tem resultados publicados e uma versão beta de sua plataforma disponível para uso, propondo-se a “\textit{prover um melhor entendimento de fenômenos sociotécnicos que acontecem na internet e equipar cidadãos com ferramentas para desenrolarem plenamente a complexidade de controvérsias}” \cite{noauthor_site_2013}.

Diferente dos demais estudos sobre controvérsias na Wikipédia  anteriormente citados, a Contropedia não se propões a medir quais artigos são controversos dentro de uma Wikipédia, e sim quais tópicos são controversos dentro de um determinado artigo. O índice de controvérsia da Contropedia é associado a \textit{wikilinks}\footnote{Link internos em verbetes da Wikipédia para outros verbetes dentro da enciclopédia.}, e é calculado somando edições com alguma remoção de conteúdo de frases onde o \textit{wikilink} apareça no artigo. Sendo que, caso mais de um wikilink apareça na mesma frase, o peso de controvérsia atribuído é proporcional ao número de wikilinks na frase editada \cite{borra_societal_2015}

Esse sutil deslocamento dos metadados utilizados se deve ao distanciamento do projeto dos esforços de criação de indicadores de controvérsia gerais, em direção a uma posição onde buscam apoiar quantitativamente pesquisadores/as que estejam observando detalhadamente a construção de artigos. A ferramenta se propõe então a ser um mapa que aponte indícios para pesquisadores/as que pretendam fazer investigações densas, chamando a atenção do/a pesquisador/a para tópicos dentro do artigo que ensejem um olhar mais aprofundado.

Concordamos com Erik Borra, que junto de outros pesquisadores do projeto Contropedia (\citeyear[p.196]{noauthor_site_2013}), afirma que ``\textit{reconhecer o potencial do histórico de edições da Wikipédia como provedor de insights sobre controvérsias sociais, e reconhecer que cada link em um artigo pode ser visto como um ponto focal de debate, nos permite utilizar a Wikipédia como um interessante local para mapear controvérsias}''. Esta afirmação ressona com as observações de Esteves (\citeyear[p.295]{esteves_as_2014}) segundo as quais ``\textit{até o fim do século XX, os cientistas sociais tinham que escolher entre análises quantitativas robustas que os distanciavam de seu tema de pesquisa ou análises qualitativas detalhadas que corriam o risco de perder de vista o contexto mais amplo em que seu objeto de estudo se inseria. [Os cientistas sociais] precisavam escolher entre falar muito sobre pouco ou pouco sobre muito. A difusão das ferramentas computacionais de análise de dados e a grande disponibilidade de registros deixados pelos usuários dos sistemas digitais na internet abriram novas possibilidades para as ciências sociais e tornaram viável superar ao menos em parte esse dilema}''.

É importante destacar na proposta de atuação da Contropedia a preocupação em ter a ferramenta como provocadora de insights, e não como uma instância julgadora que automaticamente decreta vereditos em nome do/a pesquisador/a. Como dito no próprio site do projeto, a ``\textit{Contropedia destaca conhecimento instável em oposição a fatos estáveis}'' (\cite{noauthor_site_2013}). Citando mais uma vez Latour, sobre a adaptação do uso de ferramentas quantitativas de ciência de dados para pesquisas CTS, “\textit{é como se houvéssemos passado da pesquisa dos matters of fact à exploração dos matters of concern”\footnote{Grifo do original.}} (\cite[p.160]{latour_cogitamus_2010}). Tomando esse ponto de vista, e tendo o devido cuidado de não buscar resolver as disputas estudadas mas orientar sua investigação, e atentando para o fato de que ``\textit{uma representação da realidade, quando transladada para um sistema de informação, é, inevitavelmente, representada em categorias sempre limitadas, previamente estabelecidas em estruturas de bancos de dados. [...] A maneira como essa classificação se dá, ou seja, a escolha das categorias para o enquadramento é uma questão importante, com efeitos para o coletivo}'' (\cite[p.8]{feitosa_cidadao_2010}), a aproximação que parecia improvável de ferramentas de análise quantitativa de dados com os Estudos CTS se torna possível.
