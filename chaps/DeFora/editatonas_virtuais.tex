
\section{As editatonas virtuais na prática}

*** Ver https://trello.com/c/AzsF0ayG/450-reaproveitar-texto-sobre-contropedia-quando-come\%C3\%A7ar-a-apresentar-dados-das-minhas-editatonas

\subsection{Escolha dos grupos}

Para a realização, das editatonas foram convidados grupos próximos à linha de pesquisa de Informática e Sociedade da COPPE/UFRJ, que estivessem mobilizados virtualmente durante a pandemia, para editarem verbetes sobre seu assunto de interesse.

Esta escolha de convidados pode ser considerada um movimento de interessamento mútuo da pesquisa. Ao mesmo tempo que essas são redes já mobilizadas e com proximidade facilitante ao pesquisador que conduziu as atividades, os assuntos de interesse desses grupos seriam expandidos na enciclopédia são caros à linha de pesquisa onde este trabalho é realizado, tornando assim a escolha destes grupos para a realização das atividades um tanto quanto óbvia e oportuna.
 Foram então realizadas três editatonas virtuais com os seguintes coletivos da UFRJ: Laboratório de Informática e Sociedade (LabIS), Laboratório de Informática para Educação (LIpE) e Núcleo de Solidariedade Técnica (SOLTEC).
 
\subsection{Seguindo as editatonas virtuais}

A primeira editatona com a metodologia virtual foi realizada com a equipe do Laboratório de Informática para Educação (LIpE), contando com a presença de 10 pessoas. Nesta atividade foram criados três grupos, focados em melhorar e escrever verbetes sobre as seguintes temáticas: "linguagem de programação Logo", "extensão universitária" e "história do LIpE".

Já a editatona realizada com o Laboratório de Informática e Sociedade (LabIS) teve a participação de 09 pessoas novas, além de dois membros do LIpE que haviam participado da primeira atividade e retornaram para seguir editando sobre a "história do LIpE". Assim, além do grupo que perdurou da atividade anterior, foram criados novos grupos para tratar dos temas "Libras Office", "bancos comunitários" e "Ilha do Fundão". Nesta atividade contamos com a participação de dois editores que haviam participado também de nossa atividade presencial com a turma de ECI em 2019.2, e puderam compartilhar valiosas comparações entre os dois modelos.

*** ….internet e máquina melhores que na UFRJ…. … mais cansativo de casa…

*** falar com Nayara e Rodrigo Palmeira e pegar mais declarações.

A terceira e última editatona virtual, realizada com o Núcleo de Solidariedade Técnica (SOLTEC), teve a participação de X pessoas e …..

A cada atividade realizada a metodologia de organização de editatonas virtuais sofreu alterações a partir do feedback dos participantes, com alterações inclusive na palestra de abertura.

*** atividade 1: principal feedback: terminar no horário. Um grupo também relatou que se atrapalhou quando cada um pegou uma parte do verbete para escrever separado, e melhorou quando se focaram todos na mesma seção de cada vez.

*** atividade 2: mais foco em edição na palestra de abertura.

*** narrar descoberta da "propaganda no Logo" e tentativa de interação pela PD.

*** narrar remoção do verbete sobre o Lipe.

*** Mostrar prints de verbetes

*** Números gerais. Pegar últimos parágrafos do arquivo estudosquantitativos.tex para justificar o uso de ferramentas quantitativas para análise das editatonas.

*** Apresentar quantos conteúdos de nossas editatonas ficaram no ar após X tempo.

*** Falar quantos acessos nossas edições tiveram, e qual a projeção de acessos em Y tempo.

*** Trazer fala de algum usuários específicos que tenham passado por alguma situação interessante para fechar.