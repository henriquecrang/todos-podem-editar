\subsection{O autor desta dissertação pode editar?}

\singlespacing
\begin{flushright}

\textit{``Tudo é uma questão de manter}

\textit{A mente quieta}

\textit{A espinha ereta}

\textit{E o coração tranquilo''}

Walter Franco

\end{flushright}
\doublespacing

Para proporcionar ao leitor uma apreciação situada deste trabalho, entendo ser importante explicitar aqui minha trajetória com a Wikipédia e sua comunidade. Sou militante do movimento software livre desde 2003, participando de diversas comunidades reunidas em torno de causas ligadas ao que chamamos de cultura livre, tanto a do desenvolvimento de software como a da evangelização de novos/as entrantes. Dentro destes movimentos, sempre foi corriqueira a prática de utilizar ferramentas wiki para documentar nossas atividades e construir \textit{sites} de forma colaborativa.\footnote{O TWiki, sistema mais popular de gestão de wikis do início do século teve sua primeira versão lançada em 1998 REFERÊNCIA: https://twiki.org/ .}

Em meados da década 2000 conheci a Wikipédia e, já habituado a ferramentas wiki, passei instintivamente a realizar edições pontuais como usuário anônimo. Em toda minha experiência com wikis, apenas tive usuário registrado naquelas em que a edição era restrita a cadastrados, pois não via sentido em criar uma conta em uma wiki que qualquer um poderia editar.

No dia 27 de Fevereiro de 2009 passei por uma surpresa. Ao acessar o site da Wikipédia vi em destaque no topo da página "\textit{Você tem uma mensagem nova}", e tive certeza de que se tratava de um bug da plataforma. Afinal, eu não estava logado, então não seria possível me enviarem mensagens. Decidido a investigar o bug cliquei no link, e, para meu espanto, a mensagem realmente era sobre uma edição feita por mim! Neste momento aprendi duas coisas: que a Wikipédia trata seus usuários anônimos pelo IP de conexão\footnote{Essa caraterística do MediaWiki será densamente explorada mais a frente na pesquisa.}, permitindo, ainda que de forma precária, rastreá-los minimamente; e que é importante ter um usuário cadastrado mesmo em uma wiki em que qualquer um pode editar, pois isso facilita a troca de mensagens com outros/as usuários/as editando o mesmo artigo que eu porventura também esteja editando.

Confesso um pouco de desconforto inicial com essa funcionalidade de mensagens associadas a usuários do MediaWiki. Outras ferramentas gestoras de wikis mais populares da época, como o TWiki e o TikiWiki, forneciam “apenas” páginas de discussão associadas a cada artigo criado, e não aos usuários. Assim, os debates giravam sempre em torno de uma página específica, e não existia a possibilidade enviar uma "mensagem direta" para um usuário. De cara imaginei que essa funcionalidade de mensagem direcionada a um usuário poderia deslocar discussões das páginas onde elas deveriam acontecer com mais visibilidade\footnote{Essa preocupação reaparecerá e será detalhada no capítulo 3.} além de também personificar as contribuições de uma maneira que me parecia não harmônica com a filosofia dos movimentos livres. Mas, como estava pessoalmente disposto a engajar-me com mais comunidades de produção de conteúdos, para além das comunidades de software, e percebendo a crescente importância da Wikipédia, dei o braço a torcer e criei meu usuário.

Desde então, meu usuário HenriqueCrang realizou 3368 edições nas wikis do Movimento Wikimedia, sendo 80\% delas na Wikipédia em português. Destas, 83,2\%, ou 2.182 edições\footnote{Dados de 03 de março de 2020, acessados em https://xtools.wmflabs.org/ec/pt.wikipedia.org/HenriqueCrang .}, foram feitas no chamado “domínio principal” da Wikipédia, que engloba as páginas com os verbetes enciclopédicos\footnote{A separação das Wikipédias em domínios será explicada no próximo capítulo.}.

Através de meu engajamento com redes de cultura livre, e curiosamente não pelas páginas da Wikipédia\footnote{Em alguns momentos da história os/as voluntários/a engajados em \textit{outreach} no Movimento Wikimedia no Brasil não estiveram próximos das atividades na Wikipédia em português. \textit{``Outreach''} é uma palavra em inglês que significa atividades extramuros, de extensão. Utilizaremos o termo em inglês pois a comunidade lusófona de wikipedistas utiliza-o corriqueiramente. Estas dinâmicas e separações serão melhor trabalhadas no capítulo 3.}, tive contato com o Movimento Wikimedia Brasil, que realizava atividades de \textit{outreach} para promoção das wikis, e passei a participar e organizar atividades no Rio de Janeiro. Em 2012, fiquei sabendo que a Wikimedia Foundation planejava abrir um escritório no Brasil\footnote{https://www1.folha.uol.com.br/fsp/tec/34745-wikipedia-abrira-seu-1-escritorio-no-brasil.shtml} e, ao final deste ano, candidatei-me e fui aprovado para trabalhar diretamente para a Fundação como “analista de dados e experimentos”. A ideia de abrir um escritório no Brasil acabou não vingando, mas tivemos uma equipe de quatro pessoas trabalhando no chamado “Programa catalisador do Brasil”, que buscava aumentar o número de editores ativos nos projetos Wikimedia no país, com especial foco na Wikipédia em português\footnote{Paralelamente foram realizados projetos similares na Índia e no MENA (sigla em inglês para Oriente Médio e Norte da África), regiões que foram mapeadas pela Fundação como apresentando uma baixa relação de editores/as\textbf{/}leitores/as, apresentando assim potencial para um rápido engajamento de novos/as voluntários/as.}.

Vale citar aqui que a Wikimedia Foundation tem como hábito criar um/a usuário/a “de trabalho” para seus funcionários, diferente de seu/sua usuário/a como voluntário/a. Como forma de separar o que era atuação da Fundação (que não financia diretamente a produção de conteúdo) e atuação voluntária, os/as funcionários/as que quisessem contribuir com as páginas de conteúdo deveriam fazê-lo com suas contas de voluntário/a, enquanto deveriam utilizar a conta institucional (normalmente com o nome terminado em “(WMF)” facilitando a identificação pela comunidade) para debater atividades nas quais estivesse atuando remuneradamente pela Fundação. Com meu usuário “HAndrade (WMF)” realizei mais 1612 edições, sendo a maioria delas no Meta Wiki e 36\% na Wikipédia em português.\footnote{Dados disponiveis em https://xtools.wmflabs.org/ec/pt.wikipedia.org/HAndrade\%20(WMF) , acessados em 03 de março de 2020.} Destas edições, 11 foram feitas por distração em verbetes, acreditando estar logado com a conta de voluntário.\footnote{É comum funcionários/as da WMF trabalharem com dois navegadores abertos, mantendo em um logada sua conta institucional e no outro sua conta de voluntário.}

Nestes anos em que trabalhei para a Fundação, atuei tanto auxiliando as comunidades lusófonas a desenvolver análises de dados para se conhecerem e testarem hipóteses, como também engajando voluntários/as técnicos/as para o desenvolvimento de ferramentas demandadas pelas comunidades locais. Estive assim imerso nas controvérsias da Wikipédia em português com disponibilidade, intensidade e dedicação muito maiores do que nos tempos de voluntário. Essa vivência me permitiu conhecer densamente o movimento e seus softwares, estruturas de dados, espaços de decisão e dinâmicas de governança. Tal experiência prévia serviu-me como um enorme e valioso mapa para navegar pelos espaços necessários para realizar essa pesquisa, que com certeza não teria sido desenvolvida da mesma maneira por um/a pesquisador/a recém chegado/a ao mundo Wikimedia.