\section{Estratégias para recepção de novatos.}

Uma das principais formas que o Movimento Wikimedia tem para engajar e trazer novos/as usuários/as à comunidade são as editatonas. Realizadas por todo o mundo, são eventos onde pessoas se reúnem para editar sobre um mesmo tema de interesse. Como definida pela própria enciclopédia, no verbete ``Maratona de edição'' na Wikipédia em português, uma editatona é um evento ``\textit{durante o qual editores se reúnem para editar e melhorar um tema ou tipo específico de conteúdo, geralmente incluindo um treinamento em edição básica para novos editores. A palavra é uma combinação das palavras ``editar'' (\textit{edit}) e ``maratona'' (\textit{marathon})}'' \citewiki{ptwiki_maratona}.

Existem variações em seu formato, mas normalmente a atividade começa com um (ou mais) editor/a(es/as) experiente(s) apresentando o funcionamento da enciclopédia e introduzindo suas políticas editoriais. Em sequência, os/as demais participantes criam uma conta de usuário/a e começam a escrever, individualmente ou em grupo, verbetes de seu interesse. Nesta fase, o/a(s) usuário/a(s) experiente(s) atua(m) como tutor/a(es/as), tirando dúvidas dos/as novatos/as que apareçam durante o processo de edição.

Existem também algumas editatonas que funcionam como ``forças-tarefas'' de usuários experientes melhorando verbetes sobre um determinado assunto focal, como por exemplo ``patrimônio natural brasileiro''\footnote{https://pt.wikipedia.org/wiki/Wikipédia:Edit-a-thon/Atividades\_em\_português/Wiki\_Loves\_Earth\_Brasil\_2015 , acessada em 19 de março de 2020.} ou ``eleições no Brasil''\footnote{https://meta.wikimedia.org/wiki/Programa\_Catalisador\_do\_Brasil/2013-2014/Micro-subsídios/Solicitação/Wikitona\_Eleições\_2014 , acessada em 19 de março de 2020.}. Estas atividades dispensam a explanação inicial, e, os/as presentes, editores/as já experientes, partem rapidamente para a divisão de tarefas e escrita de conteúdos. Habitualmente, essas forças tarefas são realizadas online, e a maioria das editatonas presenciais é direcionada em apresentar o mundo wiki a novos/as editores/as a partir de assuntos que sejam de seu interesse.

No início de 2020, a página para divulgação de editatonas da Wikipédia em português já contava com 120 eventos cadastrados desde 2013, sendo 115 deles (98,5\%) realizados no Brasil (\citewiki{ptwiki_edit_a_thon_atividades_portugues}). Já na página para editatonas da Wikipédia em inglês, estavam mapeados 121 eventos, com o primeiro datando de janeiro de 2011 e o último de maio de 2018 (\citewiki{enwiki_how_run_edit_a_thon}), indicando que provavelmente essa comunidade passou a registrar seus eventos realizados no último ano e meio em outro lugar. Já a Wikipédia em espanhol apresentava 92 eventos (\citewiki{eswiki_editaton}), mas aparentemente México, Argentina e Espanha, pólos movimentados de atividade wiki, estão com seus números desatualizados. Terminando nosso passeio pelas páginas sobre editatonas das maiores Wikipédias do mundo, encontramos 60 atividades mapeadas na versão francófona (com destaque para um evento realizado recentemente em Guiné, o primeiro fora do eixo França-Canadá) (\citewiki{frwiki_journées_contributives}) e 83 na enciclopédia em alemão (\citewiki{dewiki_edit_a_thon}).

Se por um lado nossa breve busca não conseguiu números precisos sobre a quantidade de editatonas realizadas pelo mundo, entendemos que os valores encontrados já são suficientes para indicar a relevância deste tipo de evento para o Movimento Wikimedia. Eles também indicam que, além da organização destes eventos ser uma estratégia muito utilizada por todo o mundo, ela é especialmente popular no Brasil.

