\begin{foreignabstract}
It is said that Wikipedia is the encyclopedia that everyone can edit. But, what can be edited? The curatorship and management of the encyclopedia are done by a community of volunteers who have their specific rules and governance policies, which are often difficult to understand and that tend to become black boxes naturalized in the operation of the encyclopedia.
We are interested in studying technological democracy in Wikipedia, with emphasis on the Portuguese version. The research intends to follow two approaches: 1) an "inside-out", following experienced users involved in the governance of the encyclopedia and in their decision-making and action processes; 2) an "outside-in," following newcomers editors and the barriers faced by them as they try to collaborate with the community, mapping the translations and negotiations done to make the newcomers' content acceptable.
Both approaches are studied by a quantitative approach, through the use of softwares that access the open databases of the encyclopedia and analyze patterns and behaviors, and also by field researches, through the organization of editathons (marathons of editions) with newcomers editors and interviews with experienced editors who are administrators of the Portuguese Wikipedia.
\end{foreignabstract}

