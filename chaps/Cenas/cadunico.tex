\subsection{Engajamentos e desengajamentos}

\textbf{Cena 4, Janeiro de 2016. Niterói, Brasil.\footnote{Cena escrita a partir de entrevista realizada com Carlos Eduardo Mattos da Cruz no Rio de Janeiro, no dia 06 de fevereiro de 2020. Quando não explicitamente atribuídas a outrem, todas as aspas desta sessão são falas do entrevistado.}}

Carlos Eduardo Mattos da Cruz, responsável pela criação de cursos para os telecentros do município\footnote{Espaços criados para o desenvolvimento de atividades educacionais complementares pelo Programa Niterói Digital, desenvolvido pela Secretaria Municipal de Educação, Ciência e Tecnologia de Niterói.}, orgulhosamente recebe o coordenador do Wiki Educação Brasil, grupo local reconhecido pela Wikimedia Foundation, para iniciar o curso de capacitação de professores/as da rede municipal no uso dos projetos Wikimedia, seguindo a proposta dos telecentros de oferecer uma ``\textit{educação contemporânea sem a ditadura do MEC}''.

O projeto desta formação foi desenvolvido por Cadunico, como Carlos gosta de se apresentar, a partir de conversas com funcionários da Wikimedia Foundation, que contaram-lhe sobre como os projetos educacionais com a enciclopédia só aconteciam em universidades, e que havia o desejo de diversificar o perfil dos editores e levar a prática de escrita wiki para outros níveis educacionais.

Cadunico decidiu tocar o projeto mesmo tendo experiências pessoais não agradáveis no mundo das wikis. Em julho de 2012, criara um verbete sobre o Gnugraf, maior evento de softwares livres para edição gráfica da América Latina, organizado por ele e por sua esposa Cléo Mattos. Pouco tempo depois, foi surpreendido com a remoção do verbete. Através do apoio de amigos que trabalhavam para o movimento e de um editor experiente, conseguiu, ``\textit{depois de muito sufoco}'', deixar no ar uma versão bem simplória do texto original. Até hoje Cadunico tem receio de editar o verbete e adicionar mais informações, para não ``\textit{chamar a atenção}'' de revisores, que podem novamente decidir apagá-lo, assim como aconteceu com sua página de usuário.

Foi em maio de 2013 que ele teve sua página de usuário apagada na Wikipédia. Em suas palavras, ``\textit{porque coloquei mais um item em uma lista de coisas que eu já tinha feito e um robô bloqueou a página}''. Curiosamente, ao observarmos o histórico da página, vemos que a ação não havia sido realizada por um robô, e sim por um dos usuários humanos mais ativos da Wikipédia lusófona. Perguntado sobre mensagens recebidas em sua página de discussão, Cadunico disse desconhecer tal ferramenta, e emendou: ``\textit{quando entro no Facebook pela primeira vez ele tem umas setinhas e umas formas de mostrar onde estão as coisas. Não me lembro de ter visto isso quando me cadastrei na Wikipédia. Só utilizei a 'caixa de areia'\footnote{Área da Wikipédia onde editores podem realizar testes.} por que eu já tinha conversado com pessoal da fundação que me falou dela. Mas ainda assim, a gente escreve lá e ninguém lê. Ninguém te dá dicas sobre o que melhorar antes de enviar conteúdo para o artigo}''.

Após frustrar-se com a Wikipédia, Cadunico ``se muda'' para o Wikilivros, projeto do Movimento Wikimedia para escrita de livros didáticos. Consegue com sucesso criar uma apostila para utilização do software de diagramação \textit{Scribus}\footnote{Apostila disponível em  https://pt.wikibooks.org/wiki/Apostila\_de\_Scribus , acessada em 30 de março de 2020.}, retomando assim sua confiança na possibilidade de trabalhar em conjunto com o movimento.

Então, em 2015, Caudnico convidou representantes do Movimento Wikimedia, que trabalhavam em conjunto com a Fundação, para serem responsáveis pela formação dos/as professores/as. Acreditou que essa chancela evitaria que situações desagradáveis como as vividas por ele na Wikipédia se repetissem. ``\textit{Eu queria botar nos telecentros o símbolo da Wikimedia, e falar que aqui é um núcleo oficial, com aula dada pelo pessoal da própria Fundação. Fizemos banners e tudo mais}''.

A meta do projeto era fazer com que professores da rede municipal de Niterói adotassem ferramentas wiki em suas práticas pedagógicas. A ideia era o/a professor/a editar em suas horas fora de sala de aula, bem como editar com seus/suas alunos/as. Para além da Wikipédia, eles/elas seriam estimulados a escrever livros didáticos em conjunto no Wikilivros e inclusive a compartilhar seus planos de aula no Wikiversidade. O projeto ainda previa, em um segundo momento, expandir a capacitação para a Secretaria de Turismo, que poderia ajudar a manter verbetes na Wikipédia sobre a cidade de Niterói.

O projeto era grandioso. Contaria com ciclos de dois anos, com um ano de capacitação seguido de um ano com os/as professores/as atuando de forma independente, e no ano seguinte um novo ciclo se iniciaria, com nova formação para reciclagem e aprofundamento. Os encontros aconteciam no Telecentro do Terminal Rodoviário João Goulart. ``\textit{Era o maior telecentro, com a melhor internet e as melhores máquinas}''. Por duas vezes na semana, a equipe de lá parava de atender os outros telecentros (o escritório de gestão de todos os telecentros do município ficava nesta unidade e sua equipe auxiliava no suporte aos telecentros menores) para se dedicar exclusivamente à capacitação Wikimedia.

``\textit{Treinamentos como este começam cheios e terminam vazios. Mas com este foi diferente. Começou cheio e terminou mais cheio ainda.}'' O boca a boca entre os/as professores/as era intenso. Todo dia um/a professor/a ligava perguntando se ainda tinha vaga, e a política da Secretaria de Educação era sempre falar que tinha, para depois se virar em acomodar todo mundo. Um dia Cadunico foi convocado pelo secretário de educação, que queria monitorar o maior problema das capacitações ofertadas no município:

- E aí, como está a evasão?

- Não teve evasão.

- Que bom, então todos os alunos que começaram terminaram?

- Mais ou menos.

- Não estou entendendo.

- Começamos com um aluno por micro, e terminamos com três.

Após um ano de treinamento, era chegado o momento dos/as professores/as criarem conteúdos sem apoio de instrutores experientes. Se aproveitando de práticas de acompanhamento da aplicação dos conteúdos dos cursos com feedback contínuo dos professores, desenvolvidas no projeto "Academia de Jogos", que ensinava a criação de softwares educacionais para crianças e professores, a Secretaria de Educação seguiria então ao longo do ano monitorando de perto a fase do projeto de edição nas wikis.

``\textit{Aconteceu com eles o que aconteceu comigo}'', conta Cadunico. ``\textit{Não era a grande maioria. \textbf{Todos estavam frustrados}. Os professores não conseguiam ter o gosto de ver seus conteúdos publicados}''. Cadunico acrescenta que um professor de história da escola municipal do Fonseca, autor de uma tese acadêmica sobre Niterói antiga, e o mais empolgado durante a capacitação, ``\textit{foi sumariamente censurado e desistiu}''.

E não foram só os professores. Novamente Cadunico se sentiu censurado pelo movimento que ele tanto admirava por ter ``\textit{a missão mais nobre do planeta: a de disponibilizar livre e gratuitamente todo o conhecimento humano}''. Enquanto os professores trabalhavam com as wikis, Cadunico resolveu começar um livro didático no Wikilivros. Um livro de pensamentos que, assim como sua página na Wikipédia, foi apagado. A justificava dada para o apagamento sentensiava que seu material não era considerado um livro didático, sem que Cadunico tivesse recebido qualquer aviso, tentativa de diálogo ou chance de defesa. ``\textit{Um livro de pensamentos não é um livro didático para ensino de português ou literatura? Isso é uma forma de censura. Não deixam explícito que tipo de livro é aceitável ou não, o que já acho errado pois para mim todo livro é aceitável, e simplesmente arrancam seu trabalho, de forma impessoal e muito fria.}''

Independente das frustrações, o projeto seguia o cronograma planejado. Depois de passado meio ano de atividades de escrita wiki sem acompanhamento dos especialistas wikipedistas, a secretaria avisou aos/às professores/as que no início do ano seguinte começaria uma nova turma do projeto, buscando levantar a demanda para realizar seu planejamento. Vários/as professores/as pediram para não acontecer uma próxima versão, solicitando que a secretaria criasse outro projeto, pois eles/as estariam sendo censurados/as e maltratados/as na Wikipédia. ``\textit{Tínhamos conseguido um pequeno exército para editar conteúdos brasileiros, e ele foi se desmotivando e se dissolvendo}''. Perante tantos retornos negativos e nenhum caso de sucesso, o projeto foi cancelado e o segundo ciclo nunca se inicio.

Em paralelo, os telecentros promovíam capacitações de LibreOffice, em parceria com Eliane Domingos e Oliver Hallot, da Associação Libre de Técnicas Abertas (ALTA), parceira local da The Document Foundation\footnote{Organização mantenedora do software LibreOffice.}. Estas capacitações, que formaram mais de 30 mil pessoas, tiveram como contrapartida da equipe de Niterói o compromisso de manter a wiki com a documentação do LibreOffice em português atualizada. ``\textit{O que foi fácil, pois os professores já haviam aprendido a utilizar o MediaWiki nos cursos da Wikimedia}''.

Os/as professores/as de inglês acompanhavam a lista de e-mails dos/as desenvolvedores/as do LibreOffice e passavam para colegas as necessidades de atualização da wiki. A versão em português da documentação era atualizada em tempo real, muitas vezes mais rápido que em inglês. Neste projeto a equipe ficou muito motivada, com professores/as editando inclusive em suas horas vagas, relatando se sentirem úteis e valorizados/as. ``\textit{No LibreOffice ninguém os revertia. Provando que o problema nos projetos Wikimedia não eram ocasionados por falta de qualidade de nossa equipe.}'' \citep{cadunico_2020}