\subsection{Para quem escrevemos?}

O presente estudo busca alcançar dois públicos-alvos de leitores/as. Pretende-se tanto atrair a comunidade de usuários/as engajados/as na governança da Wikipédia, como pessoas comprometidas com a realização de atividades para a atração de novos/as editores/as.

Ao mapear processos de tomada de decisão e expor as decisões automatizadas, tomadas por sistemas como o software MediaWiki e por robôs patrulhadores, este texto oferece para leitores/as wikipedistas experientes a oportunidade de refletir sobre a constatação de que a Wikipédia tem regras duras, muitas vezes nem conhecidas de sua própria comunidade, e que tendem a se naturalizar através de implementações em software e comportamentos precondicionados.

Olhamos para o histórico de tomadas de decisão da comunidade considerando a “\textit{indeterminação do passado, que (...) é a ideia de que o passado depende do presente e de como se reinterpreta hoje o que aconteceu ontem, sendo estas revisões verdadeiros esquemas de (re)organização do mundo}" (\cite[p.23]{feitosa_cidadao_2010}). Assim, sabendo que “\textit{estamos constantemente revisando nosso conhecimento sobre o passado à luz de novos desenvolvimentos do presente}” (\cite[40]{bowker_sorting_2007}), não tomaremos o dito senso comum da comunidade como verdade definitiva e recriaremos histórias com os enredamentos que cruzarem o caminhar da pesquisa.

Com esse movimento, espera-se oferecer aos/às membros/as da comunidade uma leitura que contribua tanto em momentos de criação de novas regras, como em discussões sobre aplicações de normas vigentes e em casos de enquadramentos/transbordamentos de exceções.

Concomitantemente, ao descrever detalhadamente a realização de editatonas, iremos agrupar práticas de organização desses eventos sugeridas pela comunidade global, expor números de resultados e descrever situações de conflito e resoluções, também fazendo desta obra uma possível referência para extensionistas do Movimento Wikimedia, envolvidos/as na organização de atividades de engajamento de editores/as novatos/as.

Também podem ser citados/as como potenciais leitores/as interessados/as nesta dissertação pesquisadores/as da área dos Estudos de Ciências-Tecnologias-Sociedades (CTS)\footnote{Optamos por esta forma de grafia para nos referenciar ao nosso campo de pesquisa, que pode ser encontrado em diferentes variações em outras obras. Latour, ao falar sobre os "\textit{Science Studies}", diz que "Nunca encontrei duas pessoas que estivessem de acordo quando ao significado do campo de estudo chamado "ciência, tecnologia e sociedade"; na verdade, raramente vi alguém que concordasse quanto ao nome ou quanto à própria existência do campo!" (\cite[p.25]{latour_ciencia_1987})}, que nas seguintes páginas encontrarão um estudo de caso que, não só lança mão de metodologias CTS, como também encontra e descreve práticas de produção de conteúdo com algumas similaridades à prática de produção científica, tão estudada por esta area. Ademais, ciente das diversas experiências desagradáveis sofridas por professores/as e pesquisadores/as ao tentarem editar a Wikipédia (são incontáveis os relatos de situações análogas às das cenas que abrem este trabalho), espera-se que este estudo possa demonstrar para leitores/as não-wikipedistas que a Wikipédia não é "gratuitamente cruel" com novatos/as, e que estes/as possam compreender as barreiras encontradas em suas tentativas de edição, e aprendam possibilidades de buscar desvios e composições em suas futuras disputas pela construção de conhecimento em wikis.