Dando sequência a apresentação do arcabouço metodológico da pesquisa, é lançada mão da Teoria Ator-Rede (TAR), utilizada para tratar dos enredamentos e estabilizações de fatos científicos e artefatos tecnológicos. Sabemos que fatos e artefatos têm sua própria dinâmica de construção (\cite{fleck_genesis_2010}), mas o caso de artigos na Wikipédia se aproxima bastante dessa dinâmica, e os estudos sobre a construção de textos científicos caem como uma luva para estudos de escritas enciclopédicas. Como diz Esteves (\citeyear[p.102]{esteves_as_2014}), “\textit{a Teoria Ator-Rede propôs definir a factualidade de uma alegação não em termos de uma suposta veracidade intrínseca, mas nos termos de sua resistência aos ataques que ela venha a sofrer. A analogia com a Wikipédia é clara. Latour parecia descrever a verificabilidade nessa passagem de Ciência em ação: ‘E a que resiste [a realidade]? Aos testes de força. Se, numa dada situação, nenhum dissidente for capaz de modificar a forma de um novo objeto, então é isso, é realidade, ao menos enquanto os testes de força não forem modificados’ (\citep[p.93]{latour_ciencia_1987})}”. Neste caso, ou seja, uma vez que cessaram provisoriamente as controvérsias, os/as especialistas e os laboratórios não estão imersos nas controvérsias pela escrita da realidade, mas os/as editores/as da enciclopédia assumem o papel de seus porta-vozes e a disputa, apesar de deslocada, segue dinâmicas similares.

Assim como o fazer da ciência, a escrita da Wikipédia também segue um “\textit{caminho muito estranho porque é invisível quando tudo vai bem}” (\citep[p.44]{latour_cogitamus_2010}). Porém, quando existem divergências entre editores/as, suas redes de aliados que sustentam verbetes instáveis tornam-se visíveis, e publicações científicas, relatórios da ONU, notícias de jornais e demais fontes são arregimentadas por wikipedistas para defender uma versão do texto do verbete. A busca por aliados/as mais poderosos/as para reforçar uma posição dá a impressão de se assemelhar ao que ocorre na tecnociência a tal ponto que a seguinte passagem de Latour (\citeyear[p.48]{latour_cogitamus_2010}), sobre a estrutura textual de publicações científicas poderia muito bem ter sido escrita sobre verbetes da Wikipédia: “\textit{a presença ou ausência de referências, citações e notas de rodapé é um sinal tão importante de que o documento é ou não sério que um fato pode ser transformado em ficção ou uma ficção em fato apenas com o acréscimo ou a subtração de referências”}.

Nos moldes de um “A vida da Wikipédia como ela é", esta pesquisa se apoiará sempre que possível em relatos de casos concretos do cotidiano da enciclopédia que demonstrem a não trivialidade de enquadramentos e fronteiras estanques, colocando uma lupa sobre situações de panes, transbordamentos e recomposições. Pois afinal, são nesses momentos em que as caixas-pretas tornam-se nada óbvias. O seu processo de estabilização (ou não), que permitirá a existência de momentos posteriores estáveis de funcionamento, é exatamente o ponto de interesse da pesquisa. Como ensina Feitosa (\citeyear[p.9]{feitosa_cidadao_2010}), "\textit{um fazer ou estudar tecnologia comprometido em evidenciar as decisões tecnopolíticas relevantes para certos coletivos, deve tentar entender e explicar a relação entre artefatos e esses coletivos, ou seja, deve tentar explicar como as coisas ditas técnicas e as demais entidades (humanas e não humanas) se relacionam, com o fim, inclusive, de indicar caminhos a serem seguidos ou evitados no fazer tecnologia}".
