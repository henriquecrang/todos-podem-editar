\begin{abstract}
    
\textit É dito que a Wikipédia é a enciclopédia que qualquer um pode editar. Mas o que pode ser editado? A curadoria e a gestão da enciclopédia são realizadas por uma comunidade de voluntários/as, que possuem regras específicas e políticas de governança muitas vezes complexas de serem compreendidas, e que tendem a não ser questionadas, naturalizando-se como caixas-pretas da operação da enciclopédia.
Estudaremos na presente dissertação as dinâmicas de governança na Wikipédia, com ênfase em sua versão lusófona atuando em duas frentes: 1) ``de dentro para fora'', seguindo usuários/as experientes envolvidos/as na governança da enciclopédia e em seus processos de decisão e ação; 2) ``de fora para dentro'', seguindo editores/as recém-chegados/as e as barreiras enfrentadas por eles/as enquanto tentam colaborar com a comunidade, mapeando as traduções e negociações feitas para tornar o conteúdo dos/as novatos/as aceitável pela comunidade.
Ambas as frentes são estudadas por uma abordagem quantitativa, através da utilização de softwares que acessam os bancos de dados abertos da enciclopédia, e analisam padrões e comportamentos, como também por estratégias qualitativas; através de entrevistas com editores/as experientes/as que são administradores/as da Wikipédia lusófona; e através da organização de editatonas (maratonas de edição) com editores/as novatos/as em distintos contextos sociais.
\end{abstract}

