\subsection{De dentro para fora e de fora para dentro}

A trajetória de nossa pesquisa começou em 2017, com o plano de dar sequência a um dos trabalhos futuros propostos pela tese de Bernardo Esteves\footnote{\textit{(...)os resultados deste trabalho também poderiam ser explorados com a cartografia de controvérsias, método de visualização das forças em oposição nas controvérsias proposto por pesquisadores alinhados com a Teoria Ator-Rede (\cite{venturini_diving_2010};\citeyear{venturini_building_2012}). As duas abordagens – um tratamento estatístico de maior escopo e a visualização com a cartografia de controvérsias – são duas perspectivas possíveis de desdobramento deste trabalho no futuro em parceria com outros pesquisadores.}" (\cite[p.296]{esteves_as_2014})} e investigar ferramentas de mapeamento e cartografia de controvérsias na Wikipédia. Em nosso esforço inaugural neste caminho foi produzido um primeiro artigo, ``\textit{Controvérsias na Wikipédia lusófona: pode o olhar CTS ser apoiado por ferramentas quantitativas?}'', apresentado no \textit{International Wikipedia Scientific Conference} (IWSC), que se propôs a ``\textit{revisar esforços prévios de mapeamentos de controvérsias na Wikipédia e analisar o funcionamento da ferramenta Contropedia, explicitando suas funcionalidades e a utilizando para observar algumas controvérsias, a fim de entender como ferramentas quantitativas podem apoiar estudos de controvérsias que utilizem abordagens CTS}'' (\cite[p.1]{de_andrade_controversias_2017}).

Após este primeiro resultado de nosso projeto de pesquisa, passamos a estudar \textit{in loco} diferentes práticas de escrita na Wikipédia, suas barreiras e resultados obtidos, com ênfase em possíveis peculiaridades da Wikipédia em português. 

Esta nova abordagem se materializou em uma apresentação feita no 6º Simpósio Nacional de Ciência, Tecnologia e Sociedade (ESOCITE.BR), intitulada “\textit{Histórias das ciências para todos: uma prática de escrita biográfica na Wikipédia em português}”, onde é “\textit{lançado um olhar CTS sobre a escrita de conteúdos enciclopédicos relacionados aos estudos de Ciências-Tecnologias-Sociedades, descrevendo três cenas de edição da Wikipédia acontecidas no contexto de uma disciplina de pós-graduação, cada qual com objetivos, enredamentos, desvios, alistamentos e resultados bem distintos}” (\cite{andrade_historias_2017}).

No desenvolvimento das duas abordagens de pesquisa abertas foi possível observar que o onipresente discurso de que “\textit{todos podem editar}” não se mostrava na prática tão óbvio como colocado pela comunidade, e o interesse por sua exploração passou então a figurar como tema central da investigação que segue. Este movimento se soma a leitura da dissertação “\textit{O discurso do global nas comunidades de software livre: estudo de caso do WordPress}”, defendida por Rodrigo Primo no Programa de Engenheria de Sistemas e Computação da COPPE/UFRJ, em 2017, na qual problematiza o discurso de adesão e representatividade global da comunidade que desenvolve o Wordpress, CMS\footnote{Do inglês Content management system, são sistemas de publicação e gerenciamento de páginas online.} mais utilizado do mundo. Sua problematização é construída a partir do estudo das dinâmicas de funcionamento e organização interna do Wordpress. Percebemos então que nosso estudo não deveria se ocupar diretamente das controvérsias presentes na escrita de verbetes, e sim nas controvérsias predecessoras a estas, onde políticas editoriais, regras de atuação e sistemas automatizados são interessados\footnote{Utilizamos na pesquisa o termo ``interessar'' da mesma forma que Callon, significando a prática de enredar actantes de forma que os mantenham como aliados estáveis.} (\cite{callon_scallops_1986}) e posteriormente ``caixapretados''.

Assim, inspirados pelos caminhos trilhados pelos dois trabalhos apresentados, percebemos que a investigação proposta precisaria ser trilhada por duas abordagens de pesquisa distintas, uma que chamaremos ``de dentro para fora'' (ADENTROF) e a outra ``de fora para dentro'' (AFORAD).

Na abordagem ``de dentro para fora'', seguimos usuários/as experientes envolvidos/as na governança da enciclopédia e em seus/suas processos de decisão e ação, a fim de observar a materialidade das práticas que dão gênese aos metadados tão utilizados pelos/as estudiosos/as de controvérsias aqui já citados/as. São apresentados aos leitores detalhes do fazer da governança da Wikipédia em português, passando por: seus processos de criação de regras; suas estratégias para garantir seu padrão de qualidade; seu software MediaWiki e seus filtros de edição, captchas e listas brancas e negras; sua relação com as comunidades globais e sua gestão dos servidores.

Já a abordagem ``de fora para dentro'' segue editores/as recém-chegados/as à prática de escrita wikipédica, e as barreiras enfrentadas por eles/as, enquanto tentam colaborar com a comunidade. Nesta abordagem da pesquisa são mapeadas traduções e negociações feitas para tornar o conteúdo dos/as novatos/as aceitável pela comunidade. Esta abordagem da pesquisa narra o histórico das atividades de \textit{outreach} do Movimento Wikimedia, com destaque para o Programa de Educação, as parceiras GLAM\footnote{Sigla em inglês para "\textit{Galleries, Libraries, Archives \& Museums}", usada pelo movimento para se referir a projetos feitos em parceria com galerias, bibliotecas, arquivos e museus.} e as maratonas de edição. Por fim, realizamos nossas próprias editatonas, para seguirmos bem de perto a dinâmica de usuários/as novatos/as participando destes eventos e editando em distintos contextos.

Ambas as abordagens foram apoiadas tanto por uma abordagem quantitativa, através da utilização de softwares que acessam os bancos de dados abertos da enciclopédia e analisam padrões e comportamentos, como também por pesquisas de campo, através de leituras de históricos de discussão e entrevistas com editores/as experientes administradores/as da Wikipédia.

Para garantir a exequibilidade desta pesquisa, muitos elementos interessantes transbordaram para fora do quadro, pois, como nos ensina Callon (\citeyear[18]{callon_markets_1998}), ``qualquer enquadramento é necessariamente sujeito ao transbordamento''. Ao caminhar da pesquisa, escolhas foram necessariamente feitas para definir recortes em zonas cinzentas em torno de nossas bordas. Todavia, ao longo do texto existe a preocupação de apontar os movimentos em que definições claras de enquadramentos se fizeram necessárias. Desta forma, a presente pesquisa não somente se mantém leal à sua metodologia, como situa o/a leitor/a no caminho trilhado, propiciando não apenas uma leitura contextualizada e honesta, como também facilitando passos de futuros/as pesquisadores/as que venham a não só replicar os experimentos e inferências desta pesquisa como também aos que se cativem por pesquisar a Wikipédia com diferentes enquadramentos, em busca de análises distintas das aqui apresentadas.


