\section{Estratégias para recepção de novatos.}

"a ampla aceitação de artefatos e arranjos organizacionais é condição sine qua non para participar de uma comunidade de prática (Lave and Wenger 1991, Star 1996). Estranhos e pessoas de fora de um grupo têm a infraestrutura como um alvo de aprendizado. Novos participantes adquirem familiaridade com estes objetos à medida que se tornam membros do grupo. (STAR; RUHLEDER, 1996, apud BOWKER; STAR, 2007, p.35)" (FEITOSA, 2010. p. 17)

# talvez aproveitar originais da fala acima

# Buscar algo do Primo sobre frustração dos entrevistados dada a dificuldade em colaborar no WP. Citar que essa dificuldade foi observada nas falas de lá, e que iremos atrás de falas da Wikipédia organizando nossas atividades.

\subsection{Coisas dentro da Wikipédia}

Seguindo esta trilha então narramos esforços da comunidade wikipedista de melhorar a recepção de novatos, como o Teahouse (MORGAN e HALFAKER, 2018), espaço na Wikipédia anglófona que funciona como um fórum com tutores e usuários novatos, e o Snuggle (HALFAKER et al., 2014), que segundo a Wikipédia (WiKIPEDIA-PT, “Wikipédia:Snuggle”, 07/07/2014) ”é um sistema web para observação e apoio a novatos [...] desenvolvido para permitir que tutores observem as atividades de usuários recém registrados separando novatos desejáveis (boa fé e produtivo (sic)) de indesejáveis (má fé e vândalos)”, a iniciativa "Wikipedia Adventure" (NARAYAN et al., 2017) e o robô ClueBot NG.


# Dificuldade de usuários novos: vários casos de sucesso da TeaHouse na enwiki, mas na ptwiki o "Café dos novatos", que foi renomeado para https://pt.wikipedia.org/wiki/Wikip%C3%A9dia:Tire_suas_d%C3%BAvidas é muito pouco usado e nada linkado.
# Posso comparar número de uso do café na en e na ptwiki


# Falar sobre Wikipedia Adventure (NARAYAN et al., 2017)
# usar (TARABORELLI e CIAMPAGLIA, 2015) 

\subsection{Outreach}

# “Quando os de dentro saem”. Trazer essa fala do Latour para falar de outreach.
# Falar de forma global do programa de educação, atividades glam e editatonas.

Eu diria ## [[citation needed]] que GLAM e Programa de Educação são as duas grandes linhas de outreach do movimento, e a editatona é uma metodologia que é operada por esses e demais tipos de projetos.
-- Mais que isso, a Editatona junta estruturas internas na Wiki com atividades outreach. Ela une os dois mundos. Isso nos chama atenção para ela

# Falar da existência da wiki de outreach. Da tentativa de compartilhamento de experiências, práticas e resultados em um movimento extensionista global e da ferramenta de compartilhamento de Learning Patterns.
# Quando falar de Learning Patterns lembrar de problematizar o conceito de melhores práticas. “Toda prática é surpreendente”. Cukierman ( PASTEUR ? )

\subsubsection{GLAM}

# Catar alguns autores de outreach para falar sobre glam por alto.

# Trazer números. Falar sobre como muitas vezes as atividades GLAM apoiam outros projetos Wikimedia, como Commons, Wikisource e Wikidata.

# Citar o projeto 1lib1ref.
-- Falar que uma das contribuições da pesquisa foi melhoria do código do citation hunt, e sua disponibilização para pt.
--- Trazer números do citation hunt.

# Citar números dos WL* e mencionar que as ferramentas para monitoramento e avaliação foram criadas pela comunidade brasile

\subsubsection{Programa de Educação}

# sobre pré-programa de educação citar interesse da comunidade de engajar acadêmicos e pesquisa (TARABORELLI 2011)

# Falar sobre a história do programa. Quando foi criado, qual sua capilaridade. Quantos cursos em quantos países, bytes adicionados….
    • Estimula professores a proporem para seus estudantes a edição verbetes da Wikipédia como uma atividade da disciplina.
    • Fundação Wikimedia cria materiais de apoio e estimula voluntários a apoiarem as iniciativas como embaixadores.

# Sobre o programa de educação, usar (MARQUES, 2012), (CARVER et al., 2012), (ARCHUBY, 2018), (SOLER-ADILLON, 2018)

…. Programa de Educação, que aconteceu entre 2011 e 2015, onde podemos observar de forma explícita a dificuldade de professores e estudantes para criar conteúdos na Wikipédia e suas tentativas de negociação com a comunidade para a criação de estratégias que possibilitassem a permanência dos conteúdos criados pelos estudantes no projeto.(MARQUES, 2012).
Sabemos que conteúdos criados em editatonas e em salas de aulas costumam ter dificuldades para permanecer na Wikipédia. São comuns os relatos de professores e alunos desmotivados por verem suas contribuições desaparecerem do ar pouco depois de serem salvas. Os casos mais repetidamente vistos como bem sucedidos adotam estratégias para tentar mitigar essa situação, como por exemplo estimular os alunos a editar primeiro em uma página de testes, onde o conteúdo será revisado pelo professor e por outros voluntários, para só então ser adicionado à página do verbete desejado. 

Assim por exemplo foi feito por Juliana Bastos Marques, professora da UNIRIO, responsável pelo primeiro caso de instanciamento do Programa de Educação da Wikimedia Foundation no Brasil. Como relatado por ela na Revista História Hoje (2012, p. 339), "uma maneira bastante segura de evitar conflitos com wikipedistas durante a realização da disciplina é fazer que os alunos escrevam seus artigos apenas em suas páginas de teste, e que após a avaliação eles sejam corrigidos pelo/a professor e embaixadores, para que então possam ser ‘colocados no ar’. Este modelo não é consenso entre os professores que adotaram o projeto até agora, mas se provou fácil e seguro". Corroborando com nossa experiência, a autora afirma no mesmo texto que “observou-se que outros cursos já ministrados no Brasil dentro do mesmo projeto não conseguiram alcançar resultados satisfatórios no mesmo grau" (MARQUES, 2012. p. 338).

# Discussão é rica para quando for falar do programa de educação é sua extensão pro MediaWiki, da necessidade de medir, e da criação de ferramentas para gestão / monitoramento dos cursos.
- Se for falar disso, trazer fala de Zé Marcos. O SI parece feito pensando na demanda de quem está no centro de poder por dados, e não no uso cotidiano dele. Os usuários parecem "perder tempo" preenchendo campos que gerarão relatórios que não são de seu interesse. Pq afinal preencher corretamente então?
# Citar Latour sobre a discussão anterior:
"[...] construir centros implica trazer para eles elementos distantes – permitir que os centros dominem à distância –, mas sem trazê-los "de verdade" [...] Esse paradoxo é resolvido criando-se inscrições que conservarão, simultaneamente, o mínimo e o máximo possível, através do aumento da mobilidade, da estabilidade ou da permutabilidade desses elementos. Esse meio termo entre presença e ausência muitas vezes é chamado de informação. Quando se tem uma informação em mãos, tem-se a forma de alguma coisa sem ter a coisa em si. Como sabemos, essas informações (ou formas, ou formulários, ou inscrições – todas essas expressões designam o mesmo movimento e resolvem o mesmo paradoxo) podem ser acumuladas e combinadas nos centros"(LATOUR, 2000, p. 396)" p. 17

\subsection{As Editatonas}

# Falar aqui que é uma forma de atividade que perpassa atividades de educação, de glam, wiki projetos e outras iniciativas avulsas, inclusive tendo Learning Patterns "próprios", compartilhados por extensionistas de diversas frentes do movimento.

Uma das principais formas que o Movimento Wikimedia tem para engajar e trazer novos/as usuários/as à comunidade são as editatonas. Realizadas por todo o mundo, são eventos onde pessoas se reúnem para editar sobre um mesmo tema de interesse. Como definida pela própria enciclopédia, no verbete ``Maratona de edição'' na Wikipédia em português, uma editatona é um evento ``\textit{durante o qual editores se reúnem para editar e melhorar um tema ou tipo específico de conteúdo, geralmente incluindo um treinamento em edição básica para novos editores. A palavra é uma combinação das palavras ``editar'' (\textit{edit}) e ``maratona'' (\textit{marathon})}'' \citewiki{ptwiki_maratona}.
# Editatonas: falar da ambiguidade do termo em inglês e da perda na tradução.

# trazer foto de uma editatona

Existem variações em seu formato, mas normalmente a atividade começa com um (ou mais) editor/a(es/as) experiente(s) apresentando o funcionamento da enciclopédia e introduzindo suas políticas editoriais. Em sequência, os/as demais participantes criam uma conta de usuário/a e começam a escrever, individualmente ou em grupo, verbetes de seu interesse. Nesta fase, o/a(s) usuário/a(s) experiente(s) atua(m) como tutor/a(es/as), tirando dúvidas dos/as novatos/as que apareçam durante o processo de edição.

Existem também algumas editatonas que funcionam como ``forças-tarefas'' de usuários experientes melhorando verbetes sobre um determinado assunto focal, como por exemplo ``patrimônio natural brasileiro''\footnote{https://pt.wikipedia.org/wiki/Wikipédia:Edit-a-thon/Atividades\_em\_português/Wiki\_Loves\_Earth\_Brasil\_2015 , acessada em 19 de março de 2020.} ou ``eleições no Brasil''\footnote{https://meta.wikimedia.org/wiki/Programa\_Catalisador\_do\_Brasil/2013-2014/Micro-subsídios/Solicitação/Wikitona\_Eleições\_2014 , acessada em 19 de março de 2020.}. Estas atividades dispensam a explanação inicial, e, os/as presentes, editores/as já experientes, partem rapidamente para a divisão de tarefas e escrita de conteúdos. Habitualmente, essas forças tarefas são realizadas online, e a maioria das editatonas presenciais é direcionada em apresentar o mundo wiki a novos/as editores/as a partir de assuntos que sejam de seu interesse.

No início de 2020, a página para divulgação de editatonas da Wikipédia em português já contava com 120 eventos cadastrados desde 2013, sendo 115 deles (98,5\%) realizados no Brasil (\citewiki{ptwiki_edit_a_thon_atividades_portugues}). Já na página para editatonas da Wikipédia em inglês, estavam mapeados 121 eventos, com o primeiro datando de janeiro de 2011 e o último de maio de 2018 (\citewiki{enwiki_how_run_edit_a_thon}), indicando que provavelmente essa comunidade passou a registrar seus eventos realizados no último ano e meio em outro lugar. Já a Wikipédia em espanhol apresentava 92 eventos (\citewiki{eswiki_editaton}), mas aparentemente México, Argentina e Espanha, pólos movimentados de atividade wiki, estão com seus números desatualizados. Terminando nosso passeio pelas páginas sobre editatonas das maiores Wikipédias do mundo, encontramos 60 atividades mapeadas na versão francófona (com destaque para um evento realizado recentemente em Guiné, o primeiro fora do eixo França-Canadá) (\citewiki{frwiki_journées_contributives}) e 83 na enciclopédia em alemão (\citewiki{dewiki_edit_a_thon}).

Se por um lado nossa breve busca não conseguiu números precisos sobre a quantidade de editatonas realizadas pelo mundo, entendemos que os valores encontrados já são suficientes para indicar a relevância deste tipo de evento para o Movimento Wikimedia. Eles também indicam que, além da organização destes eventos ser uma estratégia muito utilizada por todo o mundo, ela é especialmente popular no Brasil.

# Para falar de editatonas citar (CAMPANY e KAHILI-HEEDE, 2018), (FARZAN et al. , 2016), (LITTLEJOHN, Allison et al., 2019), (MARQUES e LOUVEM, 2013)

