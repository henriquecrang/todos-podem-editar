\subsubsection{Autores aliados}

O presente trabalho foi desenvolvido com ferramentais metodológicos dos estudos de Ciências-Tecnologias-Sociedades (CTS), tais como a Teoria Ator-Rede, a observação de movimentos de tradução/translação, enredamentos, pontos de passagem obrigatória e estabilizações, e panes, aberturas e recomposições de caixas-pretas.

Bruno Latour é o grande guia metodológico de nossa pesquisa. Seu livro, ``\textit{Ciência em ação. Como seguir cientistas e engenheiros sociedade afora}'' é o material mais citado aqui, onde buscamos conceitos de ``\textit{caixa-preta}'', ``\textit{tradução}'', ``\textit{enredamento}'', ``\textit{ponto de passagem obrigatório}'', ``\textit{central de cálculo}'', ``\textit{porta voz}'' e demais reflexões CTS sobre o fazer de fatos e artefatos pela ciência. Desta obra nos apropriamos do entendimento de que o processo de investigação deve ser flexível, sem um grande mapa em alta resolução feito a priori para ser seguido, pois ``\textit{o equipamento necessário para viajar pela ciência e tecnologia é, ao mesmo tempo, leve e variável}'' (\cite[p.10]{latour_ciencia_1987}).

Tais conceitos, tão didaticamente bem apresentados pelo referido pesquisador francês, não são exclusivos da citada obra, e, proporcionando maior riqueza às articulações aqui propostas, são em vários momentos buscados em outras publicações, tanto de autoria do próprio Latour, com destaque para o livro \textit{Cogitamus} (\cite{latour_cogitamus_2010}), como também de demais autores CTS, como Michel Callon, John Law, Ivan da Costa Marques e Henrique Cukierman.

Em Michel Callon, trabalhamos com suas definições dos conceitos que compõem a Teoria Ator-Rede em sua sociologia da tradução, observando majoritariamente dois trabalhos: o seminal ``\textit{Some elements of a sociology of translation: domestication of the scallops and the fishermen of St Brieuc Bay}'', onde a partir de um estudo de caso do declínio da população das vieiras em St Brieuc Bay são apresentados os princípios de ``agnosticismo'', ``simetria generalizada'' e ``livre associação'', e explorados conceitos como ``interessamento'', ``porta vozes'', ``ponto de passagem obrigatório'' e ``alistamento'' (\cite{callon_scallops_1986}). E seu posterior ``\textit{Actor-network theory—the market test}'', onde detalha a Teoria Ator-Rede para explicar o funcionamento dos mercados econômicos (\cite{callon_markets_1998}).

Já de John Law utilizamos observações sobre a Teoria Ator-Rede e explicações sobre o conceito de ``tradução/traição'' presentes em ``\textit{Notes on the theory of the actor-network: Ordering, strategy, and heterogeneity}'', de 1992, onde encontramos a sintética definição do movimento de tradução\footnote{Em inglês ``translation'' apresenta ambiguidade entre "tradução" e ``translação''. Como ``\textit{toda tradução implica necessariamente em uma traição}'' (\cite{law_traduction/trahison:_2006}), se faz impossível utilizar esse conceito em português com idêntico sentido, e nesta obra em português optamos por utilizar sempre a palavra ``tradução'', mas sem perder de vista que a ``translação'' segue embutida no conceito.}, que segundo Law implica em ``\textit{transformação e possibilidade de equivalência, a possibilidade de que algo (por exemplo um ator) possa significar outra coisa (por exemplo uma rede)}''(\cite[p.187]{law_notes_1992}).\footnote{No original: ``‘Translation’ is a verb which implies transformation and the possibility of equivalence, the possibility that one thing (for example an actor) may stand for another (for instance a network)'' (\cite{law_notes_1992}) p. 187.}

Ivan da Costa Marques, um dos precursores dos estudos CTS no Brasil, é utilizado como fonte para reflexões sobre o fazer da ciência, e mais especificamente com suas discussões sobre como o ``\textit{global é o local de uma determinada rede e o que ela nos apresenta como o universal é o particular no poder}''(\cite[p.4]{da_costa_marques_historia_2016}), apresentadas em sua fala intitulada ``História das Ciências, Estudos CTS e os Brasis'', feita na abertura do Simpósio Scientiarum Historia IX, em 2016.

Já Henrique Cukierman, orientador desta pesquisa, aparece como referencial com seu livro ``Yes, nós temos Pasteur. Manguinhos, Oswaldo Cruz e a história da ciência no Brasil'', que explora o mito da fundação da tecnociência brasileira no início do século 20 (\cite{cukierman_pasteur_2007}). Também utilizamos artigos diversos do autor, escritos em parceria com outros pesquisadores, que lançam o olhar CTS a variados temas correlatos à nossa pesquisa, tais como processos de produção de softwares (\cite{cukierman_pasteur_2007} et al.), construção de projetos de inclusão digital (\cite{lima_da_2011}) e traduções feitas pelo (e a partir do) movimento Software Livre (\cite{pinheiro_free_2004}).

São também trazidas para o diálogo outras pesquisas de mestrado CTS desenvolvidas na linha de Informática e Sociedade do PESC/COPPE-UFRJ, onde o presente trabalho está sendo desenvolvido, como as de Paulo Feitosa, que estuda sistemas de informação e suas relações com a cidadania, ``\textit{pressupondo que dados, em lugar de 'intrínsecos' ao mundo que se quer representar, são da ordem do mundo que se quer construir}'' (\cite{feitosa_cidadao_2010}), de José Marcos Gonçalves, que se interessa pelas promessas das Tecnologias da Informação e da Comunicação (TICs) para solução dos problemas do Sistema Único de Saúde (SUS) (\cite{goncalves_as_2016}) e de Alberto Lima, que desenvolve uma ``narrativa sociotécnica sobre telecentros, lan houses e políticas públicas'' (\cite{lima_inclusoes_2013}).

Ainda dentre as dissertações de mestrado defendidas no PESC enredadas em nossa pesquisa, destacamos o trabalho ``\textit{O discurso do global nas comunidades de software livre: estudo de caso do WordPress}'', defendido em 2017 por Rodrigo Primo, que exerceu grande inspiração para o enquadramento e direcionamento de nossa pesquisa, pois também utiliza o olhar CTS para estudar uma comunidade colaborativa de produção de conteúdo. (\cite{primo_o_2017}).

Seguindo nos aproximando à materialidade de nosso objeto de pesquisa, trazemos para o diálogo alguns autores que estudam movimentos de cultura livre, produção de software livre e demais formas de construção colaborativa de conhecimento. Eric Reymond, com seu ``\textit{The cathedral and the bazaar}'', de 1999, é considerado um dos primeiros esforços densos de entender dinâmicas de funcionamento de comunidades de produção de softwares livres (\cite{raymond_bazaar_1999}). Comunidades essas herdeiras da filosofia proposta e implementada por Richard Stallman, que cria o movimento software livre em 1983 e a Fundação Software Livre (FSF)\footnote{No original em inglês \textit{Free Software Foundation}.} em 1985, e que aqui é referenciado na publicação "\textit{Free Software, Free Society}", uma coleção de seus escritos publicada em 2010 (\cite{stallman_free_2010}). Seguindo outros estudos de comunidades de software livre, nos encontramos com o trabalho "\textit{The Future of Research in Free/Open Source Software Development}", de Walt Scacchi, que versa sobre peculiaridades e potencialidades de se pesquisar comunidades de criação de software livre (\cite{scacchi_future_2010}), e também com a pesquisa liderada por Audris Mockus, que nos brinda com estudos de caso sobre o funcionamento de duas grandes comunidades de produção de conteúdos abertos: Apache e Mozilla (\cite{mockus_two_2002} et al.).

Estreitando ainda mais o foco das pesquisas aliadas, chegamos a trabalhos que apresentam exatamente a Wikipédia e suas comunidades como objetos de estudo. Juliana Marques, professora brasileira pioneira na adoção do Programa de Educação Wikimedia em \textit{terra brasilis}, apresenta suas experiências vividas em sala de aula e em projetos de extensão nos artigos ``\textit{Trabalhando com a história romana na Wikipédia: uma experiência em conhecimento colaborativo na universidade}'' (\cite{marques_trabalhando_2012}) e ``\textit{A Wikipédia como diálogo entre universidade e sociedade: uma experiência em extensão universitária}'' (\cite{marques_wikipedia_2013}). Ainda sobre experiências de edição da Wikipédia em sala de aula visitamos estudos de casos realizados em outros países, a saber Argentina (\cite{archuby_experiencias_2018}), Estados Unidos (\cite{carver_assigning_2012}), Espanha e Sérvia (\cite{soler-adillon_wikipedia_2018}).

Já em Jonathan Morgan, encontramos um estudo de caso da criação do Teahouse, uma experiência da Wikipédia em inglês para melhor receber seus/suas editores/as novatos/as no trabalho ``\textit{Tea \& Sympathy: Crafting Positive New User Experiences on Wikipedia}'' (\cite{morgan_tea_2013}). Dario Taraborelli, diretor de pesquisa da WMF entre 2015 e 2019, nos trás em ``\textit{Beyond notability. Collective deliberation on content inclusion in Wikipedia}'' reflexões sobre formas como a comunidade decide quais conteúdos são relevantes para a enciclopédia (\cite{taraborelli_beyond_2010}), e em ``\textit{Expert participation on Wikipedia: Barriers and opportunities}'', discute a participação de especialistas em suas áreas na escrita da Wikipédia (\cite{taraborelli_expert_2011}).

Amir Sarabadani, pesquisador iraniano que atua na \textit{Wikimedia Deutschland}, capítulo\footnote{A estrutura de capítulos do Movimento Wikimedia será explicada no capítulo 3.} do Movimento Wikimedia na Alemanha, nos apresenta esforços de automatizar a detecção de vandalismo em ``\textit{Building automated vandalism detection tools for Wikidata}'' (\cite{sarabadani_building_2017}), e, seguindo essa linha de interesse em ferramentas automatizadas, Stuart Geiger se debruça sobre a atuação dos robôs nas wikis em ``\textit{When the levee breaks: without bots, what happens to Wikipedia's quality control processes?}'', estudando um interessante caso onde o principal robô de combate a vandalismo na Wikipédia em inglês para de funcionar (\cite{halfaker_rise_2013}), e depois em ``\textit{Operationalizing Conflict and Cooperation between Automated Software Agents in Wikipedia: A Replication and Expansion of 'Even Good Bots Fight' }'', onde expande o apenas quantitativo trabalho ``\textit{Even Good Bots Fight}'' com técnicas etnográficas para estudar dinâmicas das relações entre usuários robôs dentro da Wikipédia (\cite{geiger_operationalizing_2017}).

Aaron Halfaker, pesquisador líder\footnote{No original, \textit{Principal Research Scientist}.} da equipe de plataformas de aprendizado de máquina da WMF, tem vasta obra estudando a Wikipédia, inclusive com publicações feitas em parceria com os já citados Morgan, Taraborelli, Sarabadani e Geiger, e nos trás relevantes colaborações sobre a qualidade dos verbetes, a participação de robôs no controle dessa qualidade (\cite{halfaker_bots_2012}), dinâmicas de engajamento de editores novatos (\cite{halfaker_dont_2011}), ferramentas para apoiar a recepção a estes(\cite{halfaker_snuggle:_2014}) e uso de machine learning para avaliar edições individuais (\cite{halfaker_artificial_2015}).

Por último, mas não menos importante, reforçamos aqui a relevância da, já citada anteriormente, tese de doutorado defendida por Bernardo Esteves no Programa de Pós-Graduação em História das Ciências e das Técnicas e Epistemologia (HCTE), da Universidade Federal do Rio de Janeiro, em 2014. ``\textit{As controvérsias da ciência na Wikipédia em português: o caso do aquecimento global}'' foi a grande inspiração de nosso trabalho, ao realizar uma pesquisa CTS interessada em controvérsias na construção de conteúdos que se desenrolam exatamente na Wikipédia. Sua conclusão que ``\textit{de forma geral, a Wikipédia age como um porta-voz da ciência e renova a profissão de fé do enciclopedismo moderno na razão, mas revela-se também pouco aberta a outras formas de conhecimento}'' (\cite{esteves_as_2014}), é um dos catalisadores para a pergunta central de nossa investigação: afinal, "Qualquer um/a pode editar a Wikipédia", mas o que pode ser editado?