\section{Cenas corriqueiras inspiram projeto de pesquisa}

As cenas que acabaram de ser apresentadas ilustram um distanciamento entre o onipresente discurso de que a Wikipédia é a ``enciclopédia que qualquer um/a pode editar'' e a prática que emerge quando os ``qualquer uns/umas'' de fato se aventuram a criar conteúdos lá. Invariavelmente eles/as acabam se deparando com barreiras não óbvias, tais como regras editoriais complexas, robôs reversores, revisores apressados, filtros de edição ou administradores/as impacientes.

Cotidianamente a Wikipédia convive com esta dicotomia entre ser aclamada como o maior projeto de produção compartilhada de conhecimento da história da humanidade, onde ``todos podem editar'' e ser acusada de não estar aberta para receber colaborações que não estejam perfeitamente enquadradas em um determinado padrão.

São incontáveis os casos de pessoas que, mesmo familiarizadas com movimentos de cultura livre e com dinâmicas educacionais, como os/as personagens de nossas cenas, ao se aventurarem no mundo onde ``qualquer um pode editar'' encontram barreiras relacionadas a ``o que pode ser editado''. Ao mesmo tempo, wikipedistas que propagam o discurso de que ``qualquer pode editar'', e se orgulham desta abertura de seu projeto, parecem não enxergar estas barreiras, que para editores/as experientes teriam soluções óbvias.

Assim sustentamos a motivação desta pesquisa em estudar as controvérsias em torno do discurso exultante de que ``\textit{qualquer um/a pode editar a Wikipédia}''. Inspirados por cenas que insinuam a não obviedade do discurso buscamos neste trabalho abrir caixas-pretas, entendendo as dificuldades e barreiras encontradas por novos/as editores/as que buscam editar a Wikipédia em português, e como regras se naturalizam no cotidiano da operação da enciclopédia, tanto em forma de comportamentos como de softwares.

Nossa pesquisa é focada na versão em português da Wikipédia. Está informação, que pode parecer trivial para alguns/algumas, é relevante de ser explicitada pois apesar de muito estudada, a Wikipédia vê a maioria dos/as pesquisadores/as que se aproximam dela interessados em sua original e maior versão; a desenvolvida em língua inglesa. 

É muito difícil, para não dizer impossível, saber quantas pesquisas já foram desenvolvidas sobre a Wikipédia. Um esforço feito no início da década pelo projeto Wikipapers, e desatualizado desde 2013, tentou desbravar tal empreendimento. Em sua última atualização, o projeto havia mapeado 2588 trabalhos. Destes, apenas 1,7\% do total eram escritos em língua portuguesa (\cite[68]{esteves_as_2014}). Desde então o número de pesquisa em português sobre a Wikipédia pode ter subido um pouco, motivado pela realização de duas edições do Congresso Científico Brasileiro da Wikipédia (CCBWIKI) e mais duas do International Wiki Scientific Conference (IWSC), que apesar de trilíngue (inglês, espanhol e português), teve suas duas edições inaugurais realizadas em cidades lusófonas (Niterói no Brasil e Porto em Portugal), atraindo em sua maioria trabalhos escritos em português e totalizando mais de 30 trabalhos apresentados em português em todas suas edições.

Ainda assim, este volume de pesquisas lusófonas é muito pequeno perante a produção global. Uma busca rápida no Google Scholar pelo termo "Wikipedia" retorna ``\textit{aproximadamente 1.500.000 resultados}''\footnote{https://scholar.google.com.br/scholar?as\_vis=1\&q=Wikipedia\&hl=pt-BR\&as\_sdt=1,5, acessada em 06/03/2020.}, enquanto uma busca na mesma ferramenta, com o filtro ativado para trabalhos em português, retorna ``\textit{aproximadamente 59.700 resultados}''\footnote{https://scholar.google.com.br/scholar?lr=lang\_pt\&q=Wikipedia\&hl=pt-BR\&as\_sdt=1,5\&as\_vis=1, acessada em 06/03/2020.}. Se, por um lado, o Brasil é responsável pela produção de aproximadamente 2,5\% dos \textit{papers} do mundo\footnote{Número retirado de uma palestra apresentada pelo Ministro de Estado de Ciência e Tecnologia em 2011 (MERCADANTE, 2011 apud \cite{cukierman_uma_2011}) que, mesmo desatualizado, acredito ser um retrato próximo ao cenário que temos hoje.}, por outro, a participação de trabalhos lusófonos no corte buscado no Google Scholar fica em 0,04\%. Sabemos que esta busca simplória não retorna apenas trabalhos sobre a Wikipédia, e muitas publicações com outros objetos de estudo podem citar brevemente o nome "Wikipedia" e acabarem sendo somadas neste montante, mas este fenômeno de menção marginal ao termo pode acontecer em qualquer idioma, podendo ser descartado como uma das causas da disparidade. Assim, não é exagero supor que a mesma distância no volume de trabalhos produzidos entre idiomas citando o termo "Wikipedia" continuará existindo para trabalhos especificamente sobre a Wikipédia. 

Ademais, cabe também observarmos que nosso recorte linguístico tem ainda um fator transbordado que se levado em conta ampliará a distância entre conteúdos produzidos sobre cada versão da enciclopédia. Dentre os estudos realizados sobre a Wikipédia, dada a dinâmica de centros e periferias globais da ciência, é comum encontrar publicações feitas em diversos idiomas, português incluso, estudando a Wikipédia em inglês. Já a recíproca não é verdadeira. É de se esperar que a vasta maioria dos trabalhos em outros idiomas não olhem para uma versão menor e escrita em um idioma não falado nos grandes centros de pesquisa como a Wikipédia lusófona, aumentando ainda mais a distância de pesquisas produzidas entre as diferentes Wikipédias.

Por fim, apesar de una para toda a lusofonia global, cabe destacar a relevância da Wikipédia em português para o Brasil. Na Wikipédia não existem versões para as diferentes variações da língua portuguesa faladas pelo mundo e, brasileiros, portugueses e demais lusófonos convivem no mesmo projeto. Em meio toda comunidade global, a versão em língua portuguesa da enciclopédia é majoritariamente frequentada por brasileiros, que em fevereiro de 2020 foram responsáveis por 56\% dos acessos. Curiosamente, em segundo lugar no ranking de visitas ao site aparecem os Estados Unidos com 9,8\% dos acessos, a frente de Portugal com 8,4\%. Cabe destacar também que após os três primeiros colocados nenhum outro país apresenta um volume relevante de acessos, com nenhum dos demais apresentando mais de 1\% dos acessos.\footnote{Dados obtidos na ferramenta Wikimedia statistics, https://stats.wikimedia.org/\#/pt.wikipedia.org/reading/page-views-by-country/normal|table|last-month|~total|monthly} Esse volume de acessos tupiniquim, que representa em valores brutos 220 milhões de \textit{pageviews} ao mês, não deixa dúvida sobre a importância da Wikipédia em português para os/as brasileiros/as, e sustenta a relevância de sua escolha como objeto de pesquisa.

Esta realidade então nos motivou a enfocar esse estudo, feito no Brasil e em português, na Wikipédia lusófona.