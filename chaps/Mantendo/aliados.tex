\subsection{Aliados da pesquisa}

O presente estudo somente se tornou possível pelo enredamento de diversos/as aliados/as que se interessaram e tiveram agência direta em atividades imprescindíveis para o trabalho. A começar por membros/as da Wikipédia, fossem administradores/as, editores/as ou desenvolvedores/as de ferramentas, que foram entrevistados/as e compartilharam seus conhecimentos e percepções sobre o funcionamento das comunidades wikipédicas.

Posteriormente, pude contar também com a ajuda de meus/minhas colegas da linha de pesquisa em Informática e Sociedade do Programa de Pós Graduação em Engenharia de Sistemas e Computação (PESC) da COPPE-UFRJ, que mobilizaram redes engajadas em seus trabalhos de campo para realizarmos editatonas. E, por fim, com as pessoas que participaram das editatonas, dispostas a editar a Wikipédia, gerando não simplesmente conteúdos enciclopédicos, como também propiciando observações de campo e inferências em bases de dados sobre suas ações.

Seguindo os preceitos da Teoria Ator-Rede, devemos considerar não humanos/as como atores/atrizes com tanta agência e relevância como os/as humanos/as. Robôs, softwares, bancos de dados e demais elementos não são meros apetrechos neutros manuseáveis por humanos, porém participam ativamente da construção e estabilização de novos atores-redes, tanto quanto os humanos. Como propõe Arthur Leal\footnote{Professor do Instituto de Psicologia e do Programa de Pós-Graduação em História das Ciências e das Técnicas e Epistemologia (HCTE) da UFRJ.} (2015, p. 2), ``\textit{na Teoria Ator-Rede de Bruno Latour (...) o conhecimento é tomado como articulação entre diversos atores (humanos e não humanos) e não representações distanciadas e controladas entre observadores e observados}''. Humanos/as e não-humanos/as apresentam-se para a sociedade como entidades que se definem a partir de enredamentos e estabilizações precárias, e devem ser igualmente compreendidos/as como atores/as-redes que são ao mesmo tempo representantes e representados/as.

Essa pesquisa jamais teria sido possível sem a colaboração e estabilização de atores/as não humanos/as, que foram arregimentados/as para o trabalho e passaram a compor a rede que sustenta a investigação. Conforme já mencionado, esses/as atores/as são portadores/as de agência e não são neutros/as, assim seu enredamento no trabalho não é uma mera questão de retirar da prateleira os objetos desejados e os utilizar. Existem negociações feitas com cada um deles, que se apresentam como porta-vozes de determinadas redes, para que sejam levados a fazer parte da rede que sustenta o trabalho. Todo esse processo de negociação, interessamento\footnote{O termo é utilizado com o sentido de sustentar algo esperando uma vantagem. No inglês e no francês os termos “\textit{interest}” e “\textit{intéressement}” apresentam ambiguidade com a palavra “juros”, sentido esse que é necessariamente traído na tradução para o português.} e composição é, sempre que possível, explicitado e retratado ao longo do texto.

Para dar materialidade aos enredamentos anunciados, o presente trabalho realizou inferências diretas sobre as bases de dados da Wikipédia, observando metadados relacionados a atividade dos/as usuários/as no site, de onde pode-se inferir informações como números edições, históricos de reversões e perfil de atividades de usuários/as. Esta abordagem foi possível graças à característica da Wikipédia de manter praticamente todos os seus metadados históricos abertos\footnote{São mantidos em sigilo apenas os e-mails e endereços de IP dos usuários registrados. Mais informações sobre a política de privacidade do Movimento Wikimedia podem ser encontradas em https://transparency.wikimedia.org/privacy.html .}, e à disponibilização pela comunidade Wikimedia\footnote{Wikimedia é um movimento de criação colaborativa de conteúdos livres do qual a Wikipédia faz parte.} de ferramentas livres de análise de dados como o PAWS\footnote{PAWS é uma ferramenta que oferece ao usuário a possibilidade de rodar scripts Python junto aos servidores Wikimedia. Mais informações em https://www.mediawiki.org/wiki/PAWS .} e o Quarry\footnote{Quarry é uma ferramenta que permite a escrita, execução e compartilhamento de buscas SQL nos bancos de dados dos projetos Wikimedia. Mais informações em https://meta.wikimedia.org/wiki/Research:Quarry .}\footnote{Todas as queries (consultas feitas a banco de dados) realizadas durante a pesquisa estão detalhadas em um anexo deste trabalho com um link para a ferramenta Quarry, onde as mesmas podem ser novamente executadas a qualquer momento para obtenção de resultados mais recentes.}.

Seguindo os preceitos de compartilhamento de conhecimento reproduzível, todas as queries (consultas feitas a banco de dados) realizadas durante a pesquisa estão detalhadas em um apêndice com \textit{links} para a ferramenta \textit{Quarry}, onde as mesmas podem ser novamente executadas a qualquer momento para obtenção de resultados mais recentes. Desta forma, futuros/as pesquisadores/as que desejem enredar estes elementos de nossa pesquisa terão acesso facilitado à sua materialização, em uma ferramenta propícia para a realização de negociações de resultados e construções de dados.

Também cabe destacar que os dados em nossa pesquisa não foram tomados como coisas ``dadas'', encontradas neutras e prontas na natureza para só então serem vítimas das subjetividades da prática humana que adiciona parcialidades às informações produzidas a partir dele.  Não nos filiamos a teorias que acreditam em uma ``Pirâmide do Conhecimento'', com dados na base, apoiando informações, que sustentam conhecimento e no topo uma camada de sabedoria (\cite{ackoff_data_1989}). Essa imagem pressupõe que ``\textit{dados podem ser usados para criar informação, informações pode ser usada para criar conhecimento, e conhecimentos pode ser usado para criar sabedoria}'' (\cite[164]{rowley_dikw_wisdom_2007}), e assim o dado é visto como o alicerce que sustenta tudo e não é criado por nada, sendo necessariamente uma entidade pré-existente encontrada pronta na natureza.

Ao contrário, entendemos que todo dado também é um ator que se compõe a partir do acordo entre objetos, instrumentos de mensuração, sistemas padronizadores de medidas, pesquisadores, centrais de cálculo, usuários, ferramentas de armazenamento e demais elementos que aparecerão caso a caso. ``\textit{Em outras palavras, a sequência lógica tradicional 'dado, informação, conhecimento, sabedoria' é, na prática, uma construção. Sequer o 'dado' desta sequência lógica é objetivo, simplesmente oferecido ou observado. Pode-se dizer que não há nada dado, tudo é construído. \textbf{O dado não é uma dádiva}, mas sim fruto de uma construção. Desta forma, pode-se pensar em bancos de dados como bancos de negociações}'' (\cite[p.171-172]{feitosa_cidadao_2010}).

Assim, sabendo que ``\textit{o dado não é uma dádiva}'', compreendemos que quando desenvolvemos softwares que calculam indicadores, estamos a todo momento realizando escolhas, e decretando enquadramentos que necessariamente geram transbordamentos. Com isso em mente, e seguindo a preocupação apontada por Rodrigo Primo, que entende ``\textit{(...) ser fundamental falar da construção dos dados em oposição às teorias mais tradicionais do conhecimento que definem dados como sendo algo objetivo}'' (\cite[p.64]{primo_o_2017}), buscamos sempre detalhar quais dilemas foram enfrentados e quais enredamentos foram necessários para a criação de cada índice trabalhado ao longo da pesquisa, garantindo não só maior transparência ao/à leitor/a, como também de certa maneira, sujeitando-nos de forma reflexiva ao mesmo método de pesquisa que utilizamos para observar o tema investigado.

Extrair de um banco de dados algo como ``o número de usuários/as ativos/as'' pode parecer algo trivial a um primeiro olhar, mas na prática não é uma tarefa óbvia e objetiva. Afinal, quantas edições em qual período de tempo devem ser consideradas suficientes para representar a atividade de um/a usuário/a? E essas edições, podem ser feitas em qualquer local do site ou somente devem ser contadas se feitas em páginas de verbetes enciclopédicos? O tamanho das edições conta? Um/a usuário/a que fez 10 edições adicionando uma categoria em 10 verbetes diferentes em cinco minutos estaria mais ativo que outro/a que fez apenas uma edição, mas que adicionou 10 parágrafos e uma imagem a um verbete? E o que fazer com edições que tenham sido removidas? Elas não estarão disponíveis publicamente para consulta, mas o/a usuário/a que as realizou não estava ativo no momento em que as salvou?

Para o exemplo acima, de criação do indicador ``usuários ativos'', a comunidade Wikimedia, ao longo do tempo, vem estabilizando um enquadramento que o delineia, e hoje ele é aceito de forma razoavelmente estável, seguindo os critérios apontados na área de pesquisa\footnote{https://meta.wikimedia.org/wiki/Research:Wikistats\_metrics , visitado em 09 de março de 2020.} do Meta Wiki, tornando-se ponto de passagem obrigatória (\cite{latour_ciencia_1987}) para comentários e estudos sobre a atividade nas comunidades Wikimedia. Porém, algumas outras definições que nos interessam não são tão consensuais assim, como por exemplo os casos de vandalismo. Existe um contínuo debate sobre como construir esta informação nos bancos de dados, sem que até hoje uma proposta tenha enredado aliados suficientes para se estabilizar. Assim, ao ``observarmos'' este dado também o estaremos criando, reforçando algumas formas de enxergar as comunidades Wikimedia e descartando outras.

Por fim, não podemos deixar de citar como grandes aliados/as não humanos/as deste trabalho o acesso quase total e irrestrito ao histórico de discussões e espaços de tomada de decisão do movimento Wikimedia, onde são ocultadas apenas as raras edições que foram excluídas\footnote{No MediaWiki uma edição excluída some do histórico de edições. Isso somente acontece em casos que a edição possa implicar em problemas legais. Via de regra, edições indesejadas são revertidas mas não excluídas, com seu conteúdo desaparecendo da versão apresentada ao/à leitor/a na página principal do verbete, mas se mantendo disponível para consulta no histórico de edições.} e informações que poderiam ferir a privacidade de seus usuários cadastrados, como endereço IP de acesso e e-mail. Como já pontuado por \citep[p. 318]{scacchi_future_2010} ``O código-fonte, artefatos e repositórios online de projetos de software livre são fontes de dados disponíveis de maneira pública em uma escala, diversidade e complexidade que não estava previamente disponível para a pesquisa de Engenharia de Software.''. Com isso, todo o processo de construção de dados é bastante facilitado, pois o/a pesquisador/a tem independência e liberdade no ``acesso direto aos dados''. Acesso este que, obviamente, será mediado por ferramentas e escolhas pretéritas, que são  estruturantes e padronizadoras.

Mas, mesmo feita esta importante ressalva de que não existe ``acesso direto e neutro'' ao dado sem mediação, cabe pontuar que habitualmente os acessos a dados de comportamento de usuários em demais plataformas online se dá através de camadas de abstração muito mais volumosas, seja de ferramentas, bancos de dados, políticas de privacidade, NDA\footnote{Do inglês Non-Disclosure Agreement, é um acordo de confidencialidade onde o pesquisador tem limitadas as possibilidades de divulgação dos dados estudados.} assinados, acordos de cooperação, APIs\footnote{Do inglês \textit{Application Programming Interface}, são interfaces automatizadas para intercâmbio de dados entre sistemas.}, segredos industriais, etc. Assim, dada prática corriqueira de acesso nada facilitado aos dados, pesquisar a Wikipédia é a experiência mais independente de intermediários que um/a pesquisador/a interessado/a em comunidades \textit{online}, com volume massivo de usuários/as, pode dispor.